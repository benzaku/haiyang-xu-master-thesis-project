\begin{figure}[htb]
	\centering
	
	\begin{pdfpic}
	
	\begin{pspicture}
\definecolor{color202b}{rgb}{0.8,0.8,0.8}
\pspolygon[linewidth=0.04,linestyle=dashed,dash=0.16cm 0.16cm,fillstyle=solid,fillcolor=color202b](1.4848648,0.932585)(2.2545946,1.2488476)(2.88,0.69793856)(2.7549188,-0.5365059)(2.014054,-0.7711523)(1.1,0.06541331)
\pspolygon[linewidth=0.04](0.8,1.6488477)(2.4,2.2688477)(3.7,1.1888477)(3.44,-1.2311523)(1.9,-1.6911523)(0.0,-0.051152345)
\psline[linewidth=0.04](0.8,1.6288476)(2.08,0.24884766)(2.4,2.2488477)
\psline[linewidth=0.04](3.7,1.1688477)(2.08,0.22884765)(3.42,-1.1911523)
\psline[linewidth=0.04](1.9,-1.6911523)(2.08,0.20884766)(0.0,-0.051152345)
\usefont{T1}{ptm}{m}{n}
\rput(2.761455,0.19384766){$\hat{v}$}
\rput{-180.89883}(6.7641916,-2.0153627){\psarc[linewidth=0.04,arrowsize=0.05291667cm 2.0,arrowlength=1.4,arrowinset=0.4]{<-}(3.39,-0.98115236){0.93}{0.0}{77.11723}}
\usefont{T1}{ptm}{m}{n}
\rput(3.6214552,-2.0861523){$\mathcal{D}_f(\hat{v})$}
	\end{pspicture}
	\end{pdfpic} 
	\caption{a fundamental dual piece}
	\label{fig:fundDualPiece}

\end{figure}