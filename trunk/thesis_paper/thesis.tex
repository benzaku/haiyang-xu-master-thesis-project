%%% File-Information {{{
%%% Filename: template_bericht.tex
%%% Purpose: lab report, technical report, project report
%%% Time-stamp: <2004-06-30 18:19:32 mp>
%%% Authors: The LaTeX@TUG-Team [http://latex.tugraz.at/]:
%%%          Karl Voit (vk), Michael Prokop (mp), Stefan Sollerer (ss)
%%% History:
%%%   20050914 (ss) correction of "backref=true" to "backref" due to hyperref documentation
%%%   20040630 (mp) added comments to foldmethod at end of file
%%%   20040625 (vk,ss) initial version
%%%
%%% Notes:
%%%
%%%
%%%
%%% }}}
%%%%%%%%%%%%%%%%%%%%%%%%%%%%%%%%%%%%%%%%%%%%%%%%%%%%%%%%%%%%%%%%%%%%%%%%%%%%%%%%
%%% main document {{{

\documentclass[
a4paper,     %% defines the paper size: a4paper (default), a5paper, letterpaper, ...
% landscape,   %% sets the orientation to landscape
twoside,     %% changes to a two-page-layout (alternatively: oneside)
% twocolumn,   %% changes to a two-column-layout
 headsepline, %% add a horizontal line below the column title
% footsepline, %% add a horizontal line above the page footer
titlepage,   %% only the titlepage (using titlepage-environment) appears on the first page (alternatively: notitlepage)
parskip,     %% insert an empty line between two paragraphs (alternatively: halfparskip, ...)
% leqno,       %% equation numbers left (instead of right)
% fleqn,       %% equation left-justified (instead of centered)
% tablecaptionabove, %% captions of tables are above the tables (alternatively: tablecaptionbelow)
% draft,       %% produce only a draft version (mark lines that need manual edition and don't show graphics)
% 10pt         %% set default font size to 10 point
% 11pt         %% set default font size to 11 point
12pt,         %% set default font size to 12 point
%oneside, 
%openright
openright
%]{scrartcl}  %% article, see KOMA documentation (scrguide.dvi)
]{thesis}   %% \documentclass[12pt, a4paper, twoside]{thesis}

%\documentclass[defaultstyle, 12pt]{thesis}

%%%%%%%%%%%%%%%%%%%%%%%%%%%%%%%%%%%%%%%%%%%%%%%%%%%%%%%%%%%%%%%%%%%%%%%%%%%%%%%%
%%%
%%% packages
%%%

%%%
%%% encoding and language set
%%%

%%% ngerman: language set to new-german
%\usepackage{ngerman}

%%% babel: language set (can cause some conflicts with package ngerman)
%%%        use it only for multi-language documents or non-german ones
%\usepackage[ngerman]{babel}

%%% inputenc: coding of german special characters
\usepackage[latin1]{inputenc}

%%% fontenc, ae, aecompl: coding of characters in PDF documents
\usepackage[T1]{fontenc}
\usepackage{ae,aecompl}
\usepackage{rotating}
%%%
%%% technical packages
%%%

%%% amsmath, amssymb, amstext: support for mathematics
%\usepackage{amsmath,amssymb,amstext}

%%% psfrag: replace PostScript fonts
\usepackage{psfrag}

%%% listings: include programming code
%\usepackage{listings}

%%% units: technical units
%\usepackage{units}

%%%
%%% layout
%%%

%%% scrpage2: KOMA heading and footer
%%% Note: if you don't use this package, please remove
%%%       \pagestyle{scrheadings} and corresponding settings
%%%       below too.
\usepackage[automark]{scrpage2}

%%%
%%% landscape format (custom: ralf)
%%%
\usepackage{rotating}
\usepackage{float}
%%%
%%% hyphenation (custom: ralf)
%%%
\hyphenation{english}

%%%
%%% subfigures (custom: ralf)
%%%
%\usepackage{subfig}  % -> ! LaTeX Error: Option clash for package caption.

%%%
%%% compact itemize with small spacing (custom: ralf)
%%%
\newenvironment{compact_itemize}{
\begin{itemize}
  \setlength{\itemsep}{1pt}
  \setlength{\parskip}{0pt}
  \setlength{\parsep}{0pt}
}{\end{itemize}}
\newenvironment{compact_enumerate}{
\begin{enumerate}
  \setlength{\itemsep}{1pt}
  \setlength{\parskip}{0pt}
  \setlength{\parsep}{0pt}
}{\end{enumerate}}

%%%
%%% color (custom: ralf),
%%%
\usepackage{color,xcolor}

%\definecolor{darkblue}{rgb}{0,0,.5}
%\definecolor{lightblue}{rgb}{0.8,0.85,1}
%\definecolor{darkgreen}{rgb}{0,.35,0}
%\definecolor{darkgray}{gray}{.75}

%%%
%%% listings (custom: ralf)
%%%
\usepackage{listings}

\definecolor{keywordcolor}{rgb}{0,0.35,0} %dark green
\definecolor{functioncolor}{rgb}{0,0,1}  %blue
\definecolor{stringcolor}{rgb}{0.4,0,0.4}  %dark magenta
\definecolor{commentcolor}{rgb}{0,0,0}   %black




%%%
%%% listing style (custom: ralf)
%%%
\lstdefinestyle{custom}
{
    tabsize=2,
    basicstyle=\scriptsize\ttfamily, %small footnotesize scriptsize %tiny
    %backgroundcolor=\color{gray!30},
    %
    %numbers=left,
    %numberstyle=\tiny,
    %stepnumber=2, % skip every second linenumber
    %numbersep=5pt, % distance to listing
    %
    frame=single, %frame=none|leftline|topline|bottomline|lines|single|shadowbox  / frame=subset of trblTRBL
    frameround=tttt, %t = round, clockwise from top right
    framerule=0pt, %rules without width, needed for filled background
    %framexleftmargin=5mm,
    %rulesepcolor=\color{blue},
    %
    showstringspaces=false, % no special string spaces
    extendedchars=yes,
%    inputencoding=utf8
    inputencoding=latin1
}

\definecolor{pink}  {rgb}{0.67, 0.05, 0.57} % keywords
\definecolor{red}   {rgb}{0.87, 0.20, 0.00} % strings
\definecolor{green} {rgb}{0.00, 0.47, 0.00} % comments
\definecolor{violet}{rgb}{0.41, 0.12, 0.61} % classes
\definecolor{blue}  {rgb}{0.21, 0.00, 0.44} % functions
\definecolor{brown} {rgb}{0.39, 0.22, 0.13} % brown

\lstdefinelanguage{Objective-C 2.0}[Objective]{C} {
    morekeywords={id, Class, SEL, IMP, BOOL, nil, Nil, NO, YES,
                  oneway, in, out, inout, bycopy, byref,
                  self, super, _cmd,
                  @required, @optional,
                  @try, @throw, @catch, @finally,
                  @synchronized, @property, @snythesize, @dynamic},
    moredelim=[s][color{red}]{@"}{"},
    moredelim=[s][color{red}]{<}{>}
}

\lstdefinestyle{Xcode} {
    language        = Objective-C 2.0,
    basicstyle      = \footnotesize\ttfamily,
    numbers=left,
    numberstyle=\footnotesize\ttfamily,      % the size of the fonts that are used for the line-numbers
    stepnumber=1,                   % the step between two line-numbers. If it is 1 each line will be numbered
   numbersep=5pt,                  % how far the line-numbers are from the code
    identifierstyle = \color{black},
    commentstyle    = \color{green},
    keywordstyle    = \color{pink},
    stringstyle     = \color{red},
    directivestyle  =\color{brown},
    extendedchars   = true,
    tabsize         = 4,
    showspaces      = false,
    showstringspaces = false,
    breakautoindent = true,
    flexiblecolumns = true,
    keepspaces      = true,
    stepnumber      = 0,
    xleftmargin     = 0pt, 
}

\lstdefinestyle{customcpp}
{
language=C++,                % choose the language of the code
basicstyle=\footnotesize\ttfamily,       % the size of the fonts that are used for the code
%numbers=left,                   % where to put the line-numbers
numberstyle=\footnotesize\ttfamily,      % the size of the fonts that are used for the line-numbers
stepnumber=1,                   % the step between two line-numbers. If it is 1 each line will be numbered
numbersep=5pt,                  % how far the line-numbers are from the code
backgroundcolor=\color{white},  % choose the background color. You must add \usepackage{color}
showspaces=false,               % show spaces adding particular underscores
showstringspaces=false,         % underline spaces within strings
showtabs=false,                 % show tabs within strings adding particular underscores
frame=single,           % adds a frame around the code
tabsize=2,          % sets default tabsize to 2 spaces
captionpos=b,           % sets the caption-position to bottom
breaklines=true,        % sets automatic line breaking
breakatwhitespace=false,    % sets if automatic breaks should only happen at whitespace
morekeywords={attribute, varying, lowp, vec4, gl_FragColor, vec3, uniform, mat4, mat3, normalize, gl_Position},
keywordstyle=\color{pink},
escapeinside={\%*}{*)}          % if you want to add a comment within your code
}

% activate standard listing style (custom: ralf)
\lstset{style=custom}

%%%
%%% tikz (custom: ralf)
%%%
\usepackage{tikz}
\usetikzlibrary{shapes.multipart,positioning,arrows,matrix,shapes.symbols,calc,chains}

%%%
%%% custom tikz rectangle (rounded with gradient from white to blue) (custom: ralf)
%%%
\tikzstyle customrectangle=[
    % The shape:
    rectangle, rounded corners=0.4cm,
    % The size:
    minimum width=8.5cm,
    minimum height=0.8cm,
    % The border:
    very thick,
    draw=blue!50!black!50, % 50% blue and 50% black,
    % and that mixed with 50% white
    % The filling:
    top color=white, % a shading that is white at the top...
    bottom color=blue!50!black!20, % and something else at the bottom
    % Font
    %font=\small\bfseries,
    text width=8.5cm,
    text centered,
]


%%%
%%% customs (from anysl paper)
%%%
% \usepackage{color}
% \usepackage{listings}
\usepackage{array}
\usepackage{amsmath}
% \usepackage{tikz}

% \usetikzlibrary{shapes.multipart,positioning,arrows,matrix,shapes.symbols,calc,chains}

% seb's defines
\setlength{\arraycolsep}{0.5mm}
\def\<#1>{\texttt{#1}}
\def\ie{i.e.}
\def\eg{e.g.}
\def\st{s.t.}
\def\etal{et al.}
\tikzstyle cfgnodewl=[circle, draw=black, fill=white, inner sep=1pt, minimum width=4pt]
\tikzstyle cfgnode=[rectangle,draw,inner sep=4pt, minimum width=1.6cm,minimum height=12pt]
\tikzstyle cfgedge=[-stealth,thick]
\tikzstyle cfgpath=[style=cfgedge, snake=snake, segment amplitude=.4mm, segment length=2mm, line after snake=1mm]
\tikzstyle cfgbe=[style=cfgedge, dashed]

\newcommand{\loopedgeR}[2]{([xshift=3mm] #1.south) |- ([shift={(3mm, -3mm)}] #1.south east) -- ([shift={(3mm, 3mm)}] #2.north east) -| ([xshift=3mm] #2.north)}
\newcommand{\loopedgeRM}[3]{([xshift=3mm] #1.south) |- ([shift={(3mm, -3mm)}] #1.south east) -- (#2) -- ([shift={(3mm, 3mm)}] #3.north east) -| ([xshift=3mm] #3.north)}
\newcommand{\loopedgeL}[2]{([xshift=-3mm] #1.south) |- ([shift={(-3mm, -3mm)}] #1.south west) -- ([shift={(-3mm, 3mm)}] #2.north west) -| ([xshift=-3mm] #2.north)}

\newcommand{\exgraph}[3]{
	\draw (0,3.5)     node[cfgnode] (l1) { #1 };
	\draw (1.2,1.8)   node[cfgnode] (l2) { #2 };
	\draw (0,-0.1)    node[cfgnode] (l4) { #3 };
	\draw[cfgedge] (l1) -- (l2);
	\draw[cfgedge] (l2) -- (l4);
	\draw[cfgedge] ([xshift=-3mm] l1.south) -- ([xshift=-3mm] l4.north);
}

\newcommand{\dedication}{
\pagestyle{plain}
        \chapter*{}
        \addcontentsline{toc}{chapter}{}
}

%% Optional: the 'caption' package provides a nicer-looking replacement
%% for the standard caption environment. With 'labelfont=bf,'textfont=it',
%% caption labels are bold and caption text is italic.
\usepackage[labelfont=bf,textfont=it]{caption}


\usepackage{amssymb}%jetzt hab ich Quats
\usepackage{setspace}
\usepackage{graphicx}
\usepackage{booktabs}
\usepackage{algorithm}
\usepackage{algorithmic}


%%%
%%% PDF
%%%

%\usepackage{pstricks}
%\usepackage{epsfig}
%\usepackage{psfig}
%\usepackage{pst-plot}
\usepackage{ifpdf}
%\usepackage{epstopdf}
%%% Should be LAST usepackage-call!
%%% For docu on that, see reference on package ``hyperref''
\ifpdf%   (definitions for using pdflatex instead of latex)

  %%% graphicx: support for graphics
  %\usepackage[pdftex]{graphicx}
 %\usepackage{graphicx}

  \pdfcompresslevel=9

  %%% hyperref (hyperlinks in PDF): for more options or more detailed
  %%%          explanations, see the documentation of the hyperref-package
  \usepackage[%
    %%% general options
    pdftex=true,      %% sets up hyperref for use with the pdftex program
    %plainpages=false, %% set it to false, if pdflatex complains: ``destination with same identifier already exists''
    %
    %%% extension options
    backref,      %% adds a backlink text to the end of each item in the bibliography
    pagebackref=false, %% if true, creates backward references as a list of page numbers in the bibliography
    colorlinks=true,   %% turn on colored links (true is better for on-screen reading, false is better for printout versions)
    %
    %%% PDF-specific display options
    bookmarks=true,          %% if true, generate PDF bookmarks (requires two passes of pdflatex)
    bookmarksopen=false,     %% if true, show all PDF bookmarks expanded
    bookmarksnumbered=false, %% if true, add the section numbers to the bookmarks
    %pdfstartpage={1},        %% determines, on which page the PDF file is opened
    pdfpagemode=UseNone,
    colorlinks=true,
    citecolor=black,%
    filecolor=black,%
    linkcolor=black,%
    urlcolor=black
         %% None, UseOutlines (=show bookmarks), UseThumbs (show thumbnails), FullScreen
  ]{hyperref}

\usepackage{pdftricks}
  \begin{psinputs}
    \usepackage{pstricks}
  \end{psinputs}

  %%% provide all graphics (also) in this format, so you don't have
  %%% to add the file extensions to the \includegraphics-command
  %%% and/or you don't have to distinguish between generating
  %%% dvi/ps (through latex) and pdf (through pdflatex)
  \DeclareGraphicsExtensions{.pdf}

\else %else   (definitions for using latex instead of pdflatex)

  %\usepackage[dvips]{graphicx}

\usepackage{pstricks}
  \newenvironment{pdfpic}{}{}
  
  \DeclareGraphicsExtensions{.eps}

  \usepackage[%
    dvips,           %% sets up hyperref for use with the dvips driver
    colorlinks=false %% better for printout version; almost every hyperref-extension is eliminated by using dvips
  ]{hyperref}

\fi


%%% sets the PDF-Information options
%%% (see fields in Acrobat Reader: ``File -> Document properties -> Summary'')
%%% Note: this method is better than as options of the hyperref-package (options are expanded correctly)
\hypersetup{
  pdftitle={Distributed Streaming of Large Scale Geometry}, %%
  pdfauthor={Haiyang Xu}, %%
  pdfsubject={Master's Thesis, Saarland University, June 2013}, %%
  pdfcreator={}, %%
  pdfproducer={}, %%
  pdfkeywords={progressive mesh, view-dependent streaming, large scale geometry, mobile device} %%
}


%%%%%%%%%%%%%%%%%%%%%%%%%%%%%%%%%%%%%%%%%%%%%%%%%%%%%%%%%%%%%%%%%%%%%%%%%%%%%%%%
%%%
%%% user defined commands
%%%

%%% \mygraphics{}{}{}
%% usage:   \mygraphics{width}{filename_without_extension}{caption}
%% example: \mygraphics{0.7\textwidth}{rolling_grandma}{This is my grandmother on inlinescates}
%% requires: package graphicx
%% provides: including centered pictures/graphics with a boldfaced caption below
%%
\newcommand{\mygraphics}[3]{
  \begin{center}
    \includegraphics[width=#1, keepaspectratio=true]{#2} \\
    \textbf{#3}
  \end{center}
}

%%%%%%%%%%%%%%%%%%%%%%%%%%%%%%%%%%%%%%%%%%%%%%%%%%%%%%%%%%%%%%%%%%%%%%%%%%%%%%%%
%%%
%%% define the titlepage
%%%

% \subject{}   %% subject which appears above titlehead
% \titlehead{Master Thesis, Saarland University} %% special heading for the titlepage

%%% title
%\title{Automatic Shader Packetization}

%%% author(s)
%\author{Ralf Karrenberg}

%%% date
%\date{Saarbr\"{u}cken, \today{}}

% \publishers{}

%\thanks{Thanks to ALL} %% use it instead of footnotes (only on titlepage)

%\dedication{blah, blah} %% generates a dedication-page after titlepage


%%% uncomment following lines, if you want to:
%%% reuse the maketitle-entries for hyperref-setup
%\newcommand\org@maketitle{}
%\let\org@maketitle\maketitle
%\def\maketitle{%
%  \hypersetup{
%    pdftitle={\@title},
%    pdfauthor={\@author}
%    pdfsubject={\@subject}
%  }%
%  \org@maketitle
%}


%%%%%%%%%%%%%%%%%%%%%%%%%%%%%%%%%%%%%%%%%%%%%%%%%%%%%%%%%%%%%%%%%%%%%%%%%%%%%%%%
%%%
%%% set heading and footer
%%%

%%% scrheadings default:
%%%      footer - middle: page number
\setcounter{secnumdepth}{3} \setcounter{tocdepth}{2}
\pagestyle{scrheadings}

\setlength{\paperwidth}{21cm} \setlength{\paperheight}{29.7cm}


%%% user specific
%%% usage:
%%% \position[heading/footer for the titlepage]{heading/footer for the rest of the document}

%%% heading - left
% \ihead[]{}

%%% heading - center
 \chead[]{}

%%% heading - right
% \ohead[]{}

%%% footer - left
% \ifoot[]{}

%%% footer - center
% \cfoot[]{}

%%% footer - right
% \ofoot[]{}
%??????????
\lstdefinelanguage{C++A}
{
    language=C++,
    morekeywords={
    akChain, IK_MODE, SMRKinematicJoint, akArmature, SMRVector3, EE_TYPE, TrackedData
    }
}

\lstset{ %
language=C++A,                % choose the language of the code
basicstyle=\footnotesize,       % the size of the fonts that are used for the code
stepnumber=1,                   % the step between two line-numbers. If it is 1 each line will be numbered
numbersep=5pt,                  % how far the line-numbers are from the code
backgroundcolor=\color{white},  % choose the background color. You must add \usepackage{color}
showspaces=false,               % show spaces adding particular underscores
showstringspaces=false,         % underline spaces within strings
showtabs=false,                 % show tabs within strings adding particular underscores
frame=single,   		% adds a frame around the code
tabsize=2,  		% sets default tabsize to 2 spaces
captionpos=b,   		% sets the caption-position to bottom
breaklines=true,    	% sets automatic line breaking
breakatwhitespace=false,    % sets if automatic breaks should only happen at whitespace
escapeinside={\%}{)}          % if you want to add a comment within your code
}

%
% ivda-macros.tex
%

%% Jens: improve typesetting quality
\usepackage{microtype}

%% Jens: uelm enabled the \sout command for strike through text
\usepackage{ulem}
\normalem %% fix emph
%% Jens: Color include, needed for some of the macros below
%\usepackage[usenames]{color}
%% Jens: The \anonymizeForReview macro automatically replaces text with the word
%%       "anonymized" in bold gray if a "review" documentclass is choosen
%%        otherwise it's a NOOP
%\ifreviewelse{\newcommand{\anonymizeForReview}[1]{\textcolor[rgb]{0.50,0.50,0.50}{\textbf{anonymized}}}}{\newcommand{\anonymizeForReview}[1]{#1}}
%% Jens: The \TODO macro is used to flag text that should
%%       not make it into the submitted version, it is
%%       compiled to red text and should also be easy to
%%       find by a search call in the tex file before
%%       submission.
\newcommand{\TODO}[1]{\textcolor[rgb]{1.00,0.00,0.00}{\textbf{#1}}}
%\newcommand{\TODO}[1]{}
%% Jens: Helper for the \CE macro below
\makeatletter
\def\ifEmpty#1{\def\@temp{#1}\ifx\@temp\@empty}
\makeatother
%%% Jens: The \CE (=copy edit) macro should be used by the copy editor
%%% to mark changes.  The macro is used by the copy editors by not
%%% deleting the old text but putting it in the first parameter and the
%%% new text in the second, optionally a third parameter can be used
%%% for comments on the edit (if the 3rd param
%%%           is unused still write {} as otherwise latex consumes text
%%%           following the macro). if the owners approve the changes
%%%           they should only keep the second parameter if they reject
%%%           the changes they should keep the first (original text).
\newcommand{\CE}[3]{\textcolor[rgb]{0.50,0.00,0.00}{\sout{#1}}{\textcolor[rgb]
{0.00,0.50,0.00}{#2}}{\textcolor[rgb]{0.40,0.40,0.40}{\ifEmpty{#3}\else~(#3)\fi}}}
%% Jens: The the \isDraft macro to true to replace all images by
%%       (correctly sized) boxes for faster preview
\newcommand{\isDraft}{false}
%% Jens: For those that just cannot write et al. but want a macro
%%       for this purpose
\usepackage{xspace}
\def\etal{et al.\xspace}
\def\etc{etc.\@\xspace}
%% Jens: for align environment
\usepackage{amsmath,amsfonts,amssymb}
%% for using urls
\usepackage{url}
\definecolor{darkblue}{rgb}{0,0,0.75}
%% Jens: get rid of the ifpdf clash (needed for the hyperrefs below)
\makeatletter
\let\saved@ifpdf\ifpdf
\let\ifpdf\@undefined
\usepackage{ifpdf}
%\let\ifpdf\saved@ifpdf
%\makeatother
%% Jens: turn refs into links and give them a blue color (remove for print version)
%%\usepackage[colorlinks=true,linkcolor=darkblue,citecolor=darkblue,urlcolor=darkblue]{hyperref}
%% Jens: Define a new 'tinyurl' style for the package that will use a smaller font.
%%       this can be activated in the references by inserting: \urlstyle{tinyurl}
%\makeatletter

\usepackage{graphics,graphicx}
\usepackage{subfigure,epsf,epsfig,wrapfig}


% Math Commands
\newcommand{\mat}[1] {\boldsymbol{#1}} %{#1}
\newcommand{\vect}[1]{\boldsymbol{#1}}
\newcommand{\uvect}[1]{\boldsymbol{\hat{#1}}}
\newcommand{\norm}[1]{\lVert#1\rVert}
\newcommand{\abs}[1]{\lvert#1\rvert}
\newcommand{\transp}[1]{{#1}^\top}
\newcommand{\invtransp}[1]{{#1}^{-\top}}
\newcommand{\inv}[1]{{#1}^{-1}}
\newcommand{\scprod}[2]{#1\cdot#2}
\newcommand{\inprod}[2]{\left<#1,#2\right>}
\newcommand{\real}{\mathbb{R}}
\newcommand{\rthree}{\reel^3}
\newcommand{\cmplx}{\mathbb{C}}
\newcommand{\ints}{\mathbb{Z}}
\newcommand{\conj}[1]{\overline{#1}}

\newcommand{\SC}[1]{Section~\ref{#1}}
\newcommand{\SCp}[1]{Section~\ref{#1} on page~\pageref{#1}}
\newcommand{\EQWB}[1]{(Equation~\ref{#1})}
\newcommand{\EQ}[1]{Equation~\ref{#1}}
\newcommand{\EQp}[1]{Equation~\ref{#1} on page~\pageref{#1}}
\newcommand{\FG}[1]{Figure~\ref{#1}}
\newcommand{\FGp}[1]{Figure~\ref{#1} on page~\pageref{#1}}
\newcommand{\TA}[1]{Table~\ref{#1}}
\newcommand{\TAp}[1]{Table~\ref{#1} on page~\pageref{#1}}
\newcommand{\AL}[1]{Algorithm~\ref{#1}}
\newcommand{\ALp}[1]{Algorithm~\ref{#1} on page~\pageref{#1}}

\DeclareMathOperator{\sinc}{sinc}
\DeclareMathOperator{\mmid}{mid}
\DeclareMathOperator{\sincBCC}{sincBCC}
\DeclareMathOperator{\ramp}{\mathcal{R}}
\DeclareMathOperator{\boxx}{\mathcal{B}}
\DeclareMathOperator{\step}{\mathcal{H}} %{Heaviside}
\DeclareMathOperator{\tesseract}{\mathcal{T}}
\DeclareMathOperator{\hatfcn}{\Lambda}
\DeclareMathOperator{\grad}{\nabla}
\newcommand{\Fourier}[1]{\mathcal{F}\{#1\}}
\newcommand{\shah}{{\textstyle \amalg{\kern-4.pt\amalg}}}
\newcommand{\myx}[1]{{x}_#1}
\newcommand{\myy}[1]{{y}_#1}
\newcommand{\myz}[1]{\mathrm{z}_#1}
\newcommand{\myw}[1]{\mathrm{w}_#1}
\newcommand{\myxi}[1]{\vect{\xi}_#1^\perp}


%definition
\usepackage{amsthm}

\theoremstyle{plain}
\newtheorem{thm}{Theorem}%[chapter] % reset theorem numbering for each chapter

\theoremstyle{definition}
\newtheorem{defn}[thm]{Definition} % definition numbers are dependent on theorem numbers
\newtheorem{exmp}[thm]{Example} % same for example numbers

\usepackage{enumitem}% http://ctan.org/pkg/enumitem

\newcommand{\underoverrightleftarrow}[2]{\underset{#1}{\overset{#2}\rightleftharpoons}}

\renewcommand{\arraystretch}{1.5}


%%%%%%%%%%%%%%%%%%%%%%%%%%%%%%%%%%%%%%%%%%%%%%%%%%%%%%%%%%%%%%%%%%%%%%%%%%%%%%%%
%%%
%%% begin document
%%%
\begin{document}
% \pagenumbering{roman} %% small roman page numbers

%%% include the title
% \thispagestyle{empty}  %% no header/footer (only) on this page
% \maketitle

%%% start a new page and display the table of contents
% \newpage
% \tableofcontents

%%% start a new page and display the list of figures
% \newpage
% \listoffigures

%%% start a new page and display the list of tables
% \newpage
% \listoftables

%%% display the main document on a new page
% \newpage

% \pagenumbering{arabic} %% normal page numbers (include it, if roman was used above)

%%%%%%%%%%%%%%%%%%%%%%%%%%%%%%%%%%%%%%%%%%%%%%%%%%%%%%%%%%%%%%%%%%%%%%%%%%%%%%%%
%%%
%%% begin main document
%%% structure: \section \subsection \subsubsection \paragraph \subparagraph
%%%
\pagenumbering{roman}

\begin{titlepage} 

\begin{center}
    
Saarland University\\
Faculty of Natural Sciences and Technology I \\
Department of Computer Science \\[1.5cm]


{\Large Master's Thesis}\\[2.0cm]
 { 

% Title
{\LARGE \bfseries Distributed Streaming of Large Scale Geometry}\\[0.5cm]
%-}\\[0.2cm]
{\Large \bfseries
 A Framework for Geometry Streaming on Mobile Devices} \\[1.5cm]
}
    
% Author and supervisor
\begin{normalsize}
  \emph{submitted by}\\
  Haiyang Xu\\[1.2cm]
  
    \emph{submitted}\\
    10.06.2013\\[1.2cm]
    
  
  \emph{Supervisor/Advisor} \\
  Prof.~Dr.~Jens Kr\"uger\\[1.2cm]

  
  \emph{Reviewers} \\
  Prof.~Dr.~Jens Kr\"uger \\[0.1cm]
  Dr.~Tino Weinkauf \\[0.1cm]
\end{normalsize}

\end{center}


\end{titlepage}

%
% additional declaration
%

\clearpage
\thispagestyle{empty}
~
\vfill
\newpage
\begin{flushleft}
{
        \small
%
	\textbf{Eidesstattliche Erkl\"arung / Statement in Lieu of an Oath:}\\
	Ich erkl\"are hiermit an Eides Statt, dass ich die vorliegende Arbeit selbstst\"andig verfasst und keine
anderen als die angegebenen Quellen und Hilfsmittel verwendet habe.\\
	I hereby confirm that I have written this thesis on my own and that I have not used any other media or
materials than the ones referred to in this thesis.\\[\baselineskip]
	Saarbr\"ucken, June 10, 2013\\
\vspace{4cm}
	\textbf{Einverst\"andniserkl\"arung / Declaration of Consent:}\\
	Ich bin damit einverstanden, dass meine (bestandene) Arbeit in beiden Versionen in die Bibliothek der
Informatik aufgenommen und damit ver\"offentlicht wird.\\
	I agree to make both versions of my thesis (with a passing grade) accessible to the public by having
them added to the library of the Computer Science Department.\\[\baselineskip]
	Saarbr\"ucken, June 10, 2013
\vspace{3cm}
}
\end{flushleft}
\clearpage
\thispagestyle{empty}
~
\vfill



\newpage
\begin{spacing}{1.5}


\pagestyle{plain}
	\begin{quote}
        \raggedleft {\em To my parents Jianhua Xu and Min Zhang}
   	\end{quote}



\newpage
\chapter*{Abstract}
\label{chapter:abstract}

 	The abstract is the initial contact of the reader with the report. It
 	must motivate the reader to continue reading the paper, but it must
 	also archieve the appropriate expectations, such that the reader will
 	not be disappointed if the content doesn't fulfill the initial
 	promises.
 	1-2 sentences about the environment of the report: What is it
 	  all about? What is the current state of the art?
 	1-2 sentences describing the problem: What kind of problem do we
 	  want to solve, why is it important and relevant to the reader?
 	1-2 sentences about the approach or the solution strategy: How
 	  is the problem approached? What is the solution based on? If
 	  appropriate, a very short description of the solution can follow



%%% start a new page and display the table of contents
\addtolength{\textheight}{9mm}
\newpage

\tableofcontents
\listoffigures
\listoftables
\lstlistoflistings
\addtolength{\textheight}{-9mm}
\newpage
\thispagestyle{empty}
%\pagestyle{empty}
\clearpage\mbox{}\clearpage
%\tableofcontents
%\listoffigures
%\listoftables


%\newpage 
\pagenumbering{arabic}
%\pagestyle{scrheadings}


\chapter{Introduction}
\label{chapter:introduction}
\TODO{Here is just a test for literature citation. \\
\cite{Krekhov:12:MSc}, \cite{Hoppe:1996:PM}, \cite{Hoppe:1997:VRP}, \cite{Kim:2001:trulyselective}, \cite{Kim:2003:TransitiveMeshSpace}, \cite{Kim:04:VDstreaming}, \cite{Yang:2004:VDMeshTrans}, \cite{Deb:2004:DesignStreamSys}
, \cite{Callahan:2006:PVR}, \cite{Pacanowski:2008:ESS}, \cite{Cheng:2007:AMP}, \cite{Bajaj:1999:PCTriMesh}, \cite{Khodakovsky:2000:PGC}, \cite{Deb:2006:RSRT}, \cite{Hoppe:1998:EIPM}
}
\\
\TODO{introduce this paper! }
\\

With the ever fast development of modern computer science, computer graphics and visualization has become a big topic. And with the more and more advanced 3D scanner and surface reconstruction technology, people are able to get extremely large 3D model with much more detail than ever before that are scanned from real objects such as sculptures. 
And mean while, in recent years, various mobile devices (such as iPhone, iPad, Google Nexus series, etc.) with much powerful computing resource are being designed and manufactured. With faster CPU/GPU and larger memory, these hand-held devices are made possible to run graphics programs or to view 3D models. These two trends have raised many topics about graphics development on mobile devices. The visualisation of large-scale 3D model is one of the hot topics among them. 

\smallskip

In this thesis, we proposed a streaming framework for large scale geometry models. In this framework users can connect our server from a mobile device (e.g. iPad) and view the 3D geometry model they choose progressively. User can browse the model using drag-and-zoom gesture on the multi-touch screen of the device and the system will refine the corresponding part of the viewing model according to users' viewing angle. Therefore our system can provide view-dependent, selective geometry streaming. 

\smallskip
In the following paragraphs of this chapter, we will introduce in detail the motivation of the thesis in Section~\ref{section:motivation}. And in Section~\ref{section:background} we will introduce the background of this topic. And Section~\ref{section:relWork} related works in the area of this topic will be listed and discussed. 

\smallskip
And then In Chapter~\ref{chapter:BasicConcepts}, we will describe some basic theoretical and technical concepts behind the topic of this thesis including \emph{Progressive Mesh}, the \emph{Quadric Error Metrics}, \emph{View-dependent Progressive Mesh}, \etc

\smallskip
In Chapter~\ref{chapter:SystemDesign}, we will describe the overall system architecture design including the server architecture and client architecture. 

\smallskip
In Chapter~\ref{chapter:SystemImplementation}, we will describe in detail about the implementation of our geometry streaming framework. In this chapter, both server and client side implementation will be illustrated. And after that there will be a short discussion of the implementation. 

\smallskip
In Chapter~\ref{chapter:ExperimentalEvaluation} and Chapter~\ref{chapter:CollaborationWithOtherProjects} we will show the experiment result of our framework and we will also describe the use of our framework in other projects. 

In the last chapter, we will conclude this thesis. Future work will also be discussed in this chapter. 

\section{Motivation}
\label{section:motivation}
%\TODO{In this section the motivation of our project will be described.}
In this section, we will describe the motivation of this thesis. 

\smallskip
Nowadays, with the fast development of hardware and computing power, more and more complex geometry models are being used for viewing more details and better visual quality. 

\smallskip
Maybe 10 years ago, people may thought the Stanford Bunny model with almost 70,000 triangles would already enough to show its surface details. But today, with more powerful hardwares, models with much more triangles are becoming increasingly popular. For instance, models we use from \textbf{The Stanford 3D Scanning Repository\footnote{\label{S3DSR}\url{http://graphics.stanford.edu/data/3Dscanrep/}}} are almost with over 10 million triangles. People using those large models in various application such as 3D game, industrial design etc.
 
\smallskip
In the traditional local application scenario, with a desktop PC with a powerful GPU, we can view this kind of large scale geometry models easily. However, problems came out when we want to view them via internet. In the traditional local application scenario, the whole model is downloaded once and then being rendered to screen. This approach is intuitive and efficient when dealing with a model which is relatively small. But when models over hundreds megabytes are being transmitted over the internet, long waiting time becomes a problem. Under such circumstance, we will lose rendering quality and users' satisfactory as well. 

\smallskip
And meanwhile, another development trend of today's information technology is that everything is going mobilised! From the first iPhone, to Android, from the release of Microsoft's Surface tablet to the reborn of Blackberry 10. We can see that almost suddenly those handheld devices have dominated our life. You can see people using an iPad, Nexus 7 or Windows 8 table everywhere. And this also inspire us to build our project's application over a handheld device: iPad. 

\smallskip
With handheld devices, new problems are raised. There are various limitations of those handheld devices such as less powerful computing resource, smaller size of RAM, and limited network connectivity. It's already stressful for a desktop to view a large model in the traditional download-whole-once scenario, not to mention how difficult it would be for a mobile device. 

\smallskip
Therefore, intuitively \emph{streaming approach} has been raised into our mind. That is the motivation of this project. In our approach, a base, coarsen geometry model will be transmitted to the client side upon an initial connections attempt. And after that, model details will be transmitted and the client-side model will be refined on the fly. Similar to video stream on \textbf{Youtube\footnote{\label{UTUBE}\url{http://www.youtube.com/}} }. And after the refinement streaming is finished, we can get the original mesh. Users can view the model during the refinement process. 


\section{Background}
\label{section:background}
\TODO{Here we will introduce background information of this topic.}

\section{Related Work}
\label{section:relWork}
\TODO{In this section we will introduce previous research in this area. }

%Picture
%\noindent
%\begin{minipage}{\linewidth}
%\makebox[\linewidth]{%
%\includegraphics[width=1.0\textwidth]{images/morphable.pdf}}
%\captionof{figure}{MorphableUI generates user-tailored interfaces for arbitrary applications in arbitrary environments. Users are able to use all available devices to control as many applications as needed. User behavior is analyzed by the system to increase the user experience.}% only if needed
%\label{fig:morphable}
%\bigskip
%\end{minipage}



\chapter{Basic Concepts}
\label{chapter:BasicConcepts}

This section will be divided into 3 parts. In Section \ref{section:ProbStat} I will describe the problems in this thesis that I am going to solve. And In Section \ref{section:TheoConcpt} I will describe in detail the theoretical algorithms and concepts that solve the problems stated in Section \ref{section:ProbStat}. Section \ref{section:TechConcpt} introduce the techniques I used this thesis project. Further implementation details will be described in Chapter \ref{chapter:SystemDesign} and Chapter \ref{chapter:SystemImplementation}. 

\section{Problem Statement}
\label{section:ProbStat}


As the title of this thesis reads, the main contribution of our approach is the \emph{streaming} of geometries. Naturally how to achieve streaming becomes the core problem in our approach. \\
%statement of the streaming
For a precise statement of the problems, we have to initially define \emph{mesh streaming} properly. Here is a practical scenario: Suppose there is a mobile device user who would like to view a 3D model remotely stored on a server. In the traditional setup, the 3D model would be transmitted to the client side as a whole package. Once the client side get the whole model it would load it and render it to screen. This solution would be good when the model is small and the transmission of the whole model would be finished in acceptable time. While in the streaming setup, on the user have chosen a model, a rough model simplified from the original model with much smaller size and can be downloaded from client side within an acceptable time. Afterwards, details of the model would be sent to client side and the client side model will be refined on the fly until highest resolution level is reached. During the whole process the 3D model on the client side would be continuously and progressively reconstructed. Based on this scenario, we define \emph{Mesh Streaming} as follows: 
\begin{defn}
\emph{Mesh Streaming} is a process satisfying the following conditions: 
	\begin{enumerate}[label=\roman*]
		\item A rough model of small size is transmitted to the client side in the initial phase.
	  	\item Detail information of the model is transmitted after the initial phase. 
	  	\item Model in the client side can be refined and reconstructed continuously and progressively upon the details received. 
	\end{enumerate}
\end{defn}
%now we have the definition of mesh streaming, we can go further to describe the problem. 
Given the definition of \emph{Mesh Streaming}, we can get the following problems naturally: 
\begin{enumerate}[label=\roman*]
	\item \emph{Mesh representation}\\
	Since our approach is based on 3D models, a polygon mesh representation is initially needed. Therefore we have to find a way to model a 3D mesh with both efficiency and convenience. 
	\item \emph{How to produce small-sized rough models? }\\
	In the initial phase of our approach a small-sized rough mesh has to be generated from the original mesh. How to perform such kind of simplification could be a big topic. In our approach we use the \emph{Quadric Error Metric Method} which we will describe in detail in Section \ref{section:TheoConcpt}. 
	\item \emph{How to reconstruct the original mesh given the rough mesh and details? }\\
	As is stated in the definition of \emph{Mesh Streaming}, the final phase of streaming is reconstruction of the original mesh through the details streamed from the server side. And how to refine the mesh based on detail streamed from the server and finally reconstruct the original mesh becomes another problem we have to solve in this thesis. 
\end{enumerate}

Furthermore, as introduced in Chapter \ref{chapter:introduction}, our approach is providing \emph{view-dependent} streaming. Here we define the term \emph{View-dependent geometry streaming} as follows: 
\begin{defn}
	\emph{View-dependent geometry streaming} is a streaming process, in which the system responds to user's view on the geometry model and stream corresponding detail information for refinement of the model. 
\end{defn}
This definition raises another question - how to achieve view-dependent streaming? This question is related with a series of problems in data structures and algorithms. \\

In the following chapters of this thesis, we will try to answer and solve the problems raised above. 

\section{Theoretical Concepts}
\label{section:TheoConcpt}
As the problems stated in Section \ref{section:ProbStat}, in this section we will introduce the theoretical concepts which are related to the problems raised. 

\subsection{Notation}
\label{subsection:notation}
In modern computer science, models are usually represented in the form of \emph{triangular meshes}. Here we define a \emph{triangular mesh} as follows:
\begin{defn}
	A \emph{triangular mesh} is a surface consisting of a set of triangles pasted together along their edges\cite{Hoppe:1996:PM}. 
\end{defn}
A triangular mesh $M$ has three kinds of \emph{mesh elements}: vertices, edges and faces. Thus, we have the following definitions:
\begin{defn}
	\emph{Mesh Connectivity}, or topology, is the information of a mesh $M$ which describes incidence relations of mesh elements(\eg, adjacent vertices and edges of a face, \etc).   
\end{defn}
\begin{defn}
	\emph{Mehs Geometry} is the information of a mesh $M$ which specifies the position and other geometric characteristics of each vertex in $M$.
\end{defn}

\begin{figure}[htb]
	\centering
	
	\begin{pdfpic}
		\usefont{T1}{ptm}{m}{n}
		\rput(9.100586,0.14582032){$v_j$}
		\usefont{T1}{ptm}{m}{n}
		\rput(4.9911327,0.2858203){$v_i$}
		\usefont{T1}{ptm}{m}{n}
		\rput(6.448467,0.42582032){$e_{ij}$}
		\usefont{T1}{ptm}{m}{n}
		\rput(6.8691015,4.3858204){$v_0$}
		\usefont{T1}{ptm}{m}{n}
		\rput(1.3935449,3.8258202){$v_1$}
		\usefont{T1}{ptm}{m}{n}
		\rput(0.16804688,-0.81417966){$v_2$}
		\usefont{T1}{ptm}{m}{n}
		\rput(2.400576,-4.41418){$v_3$}
		\usefont{T1}{ptm}{m}{n}
		\rput(6.5692773,-4.37418){$v_4$}
		\psline[linewidth=0.04](1.6170117,3.4808204)(6.797012,4.06082)(8.697012,0.2808203)(6.797012,-3.8991797)(2.4370117,-3.7791796)(0.6970117,-0.7991797)(1.6370118,3.4608202)
		\psline[linewidth=0.04](1.6170117,3.4808204)(4.357012,0.0)(0.6970117,-0.7991797)
		\psline[linewidth=0.04](2.4370117,-3.7791796)(4.357012,0.0)(6.777012,-3.8791797)
		\psline[linewidth=0.04](8.697012,0.2808203)(4.357012,0.0)(6.797012,4.06082)
	\end{pdfpic} 
	\caption{Example: $v_i$ and $v_j$ are vertices in mesh $M$ connected through edge $e_{ij}$. And $v_i$'s neighbor vertices set $N_{v_i}$ = \{$v_0$, $v_1$, $v_2$, $v_3$, $v_4$, $v_j$\}}
	\label{fig:DefMesh}

\end{figure}

For an edge in mesh $M$ connecting two vertices $v_i$ and $v_j$, we can denote it by $e_{ij}$. And for a single vertex $v_i$, there is a set of 1-ring neighbor vertices that each of them is connected to $v_i$ with an edge. We can denote this vertex set by $N_{v_i}$. (see \FG{fig:DefMesh})

\subsection{Progressive Mesh}
\label{subsection:theoreticalPM}
In the approach proposed in this paper, \emph{progressive mesh} is used as the multi-resolution representation of a mesh. In \cite{Hoppe:1996:PM} Hoppe \etal introduced the concept of \emph{progressive mesh (PM)} based on two basic mesh transformation operations: \textbf{edge collapse} and \textbf{vertex split}. These two transformation operations are essential in the generating and reconstruction phase of a PM respectively. Hence, we will first introduce these two transformations. 

\begin{defn}
	In a triangular mesh $M$, given two vertices $v_u$, $v_t$ and the edge $e_{ut}$ connecting them, an \emph{edge collapse} transformation operation $ecol(v_s,v_u,v_t,v_l,v_r)$ collapses two vertices $v_u$ and $v_t$ connected with edge $e_{ut}$ into a new vertex $v_s$ and produces a new mesh $M'$. Here $v_l$ and $v_r$ are the two vertices in the two triangles sharing edge $e_{ut}$ which remain unchanged during the transformation. 
\end{defn}

The \emph{edge collapse} is defined above. And we can then define \emph{vertex split} as follows:
\begin{defn}
	In a triangular mesh $M$, given a vertex $v_s$ and two new vertex $v_u$ and $v_t$, a \emph{vertex split} transformation operation $vsplit(v_s,v_u,v_t,v_l,v_r)$ splits a vertex $v_s$ into two vertices $v_u$ and $v_t$ and resulting a new mesh $M'$ in which edge $e_{ut}$ connects vertices $v_u$ and $v_t$ and is the sharing edge of the two neighboring triangles: $Triangle(v_u,v_t,v_l)$ and $Triangle(v_u,v_t,v_r)$. 
\end{defn}

\FG{fig:ecol_vsplt_illustration} illustrates the $ecol$ and $vsplit$ transformation operations. Obviously they are reverse operations. The last two parameters of $ecol$ operation are $v_l$ and $v_r$ are the opposite vertices of the edge $e_{ut}$ which is to be collapsed. And in the $vsplit$ operation the last two parameters are also $v_l$ and $v_r$. However, not like those in $ecol$ operation, the $v_l$ and $v_r$ in $vsplit$ operation can be any two different vertices in the neighbor vertices set $N_{v_s} $of $v_s$. These two vertices are crucial in the $vsplit$ operation since they determine the connectivity of $v_u$ and $v_t$ with $N_{v_s}$ after $vsplit$ operation. And as described in \cite{Kim:2003:TransitiveMeshSpace}, we call the vertices $v_l$ and $v_r$ the \emph{cut vertices} of $ecol$ and $vsplit$ operations.  

\input{graphs/GraphEcolVspltillu}

In a general framework of \emph{progressive mesh}, there are two phases, the \emph{simplification phase} and the \emph{reconstruction phase}. In \emph{simplification phase} we perform $ecol$ operation on the original mesh $\hat{M}=M^n$ iteratively until a coarser mesh $M^0$ is produced. On the other hand, in \emph{reconstruction phase} the $vsplit$ operation is iteratively performed on the coarsest mesh $M^0$ which is produced in \emph{simplification phase} until the original mesh $M^n$ is reached. $n$ here indicates the number of reconstruction steps. \\

Therefore we can express the \emph{simplification phase} as follows:
$$
	(\hat{M}=M^n)\xrightarrow{ecol_{n-1}}{M^{n-1}}\xrightarrow{ecol_{n-2}}\ldots\xrightarrow{ecol_1}{M^1}\xrightarrow{ecol_0}{M^0}
$$
In each step in the \emph{simplification phase} from $M^{i+1}$ to $M^i$, we denote an $ecol$ operation of resolution level $i$ by:
$$
	ecol_i=ecol(v_{s_i},v_{t_i},v_{u_i},v_{l_i},v_{r_i}),
$$
where $i\in\{x|x\in{\mathbb{Z}}\wedge 0\le x <n\}$. By performing the $ecol$ operation until the $M^0$ is reached, a sequence of $ecol$ operations is generated: 
$$(ecol_{n-1},ecol_{n-2},\ldots,ecol_{1},ecol_{0})$$
As is stated in the previous paragraph, there is a key observation that the $ecol$ operations are invertible. In other words, in any edge collapse operation $ecol_i$, detail information $d_i$, consisting of $(v_{s_i},v_{t_i},v_{u_i},v_{l_i},v_{r_i})$,is reserved to reconstruct $M^{i+1}$ from $M^{i}$. And if we extract each detail information from the $ecol$ sequence and apply them into $vsplit$ operation in reversed order, we will get a sequence of $vsplit$ operations:
$$
	(vsplit_{0},vsplit_{1},\ldots,vsplit_{n-2},vsplit_{n-1}),
$$
where for any $0\le i < n$, a $vsplit$ can be expressed as:
$$
	vsplit_i=vsplit(v_{s_i},v_{t_i},v_{u_i},v_{l_i},v_{r_i})
$$
Therefore in the \emph{reconstruction phase} an arbitrary triangle mesh $\hat{M}$ can be represented as a simplified mesh $m^0$ together with a sequence of n $vsplit$ records:
$$
	M^0\xrightarrow{vsplit_0}M^1\xrightarrow{vsplit_1}\ldots\xrightarrow{vsplit_{n-2}}M^{n-1}\xrightarrow{vsplit_{n-1}}(M^n=\hat{M})
$$
As described in \cite{Hoppe:1996:PM}, we call $(M^0,\{vsplit_0,\ldots,vsplit_{n-1}\})$ a \emph{progressive mesh} representation of $M$. And the detail information sequence generated from the $ecol$ operations are denoted as \emph{PM sequence}.\\

Note that in a edge collapse operation $ecol_i$, the \emph{cut vertices} $v_{l_i}$ and $v_{r_i}$ are always the opposite vertices of the collapsed edge $e_{{u_i}{t_i}}$. Actually the edge collapse operation $ecol_i$ can be performed without specifying those two cut vertices because the connectivity information among the \emph{cut vertices} $(v_{l_i},v_{r_i})$ and collapsed edge $e_{{u_i}{t_i}}$ is implicitly implied in the \emph{edge collapse} operation itself. However, the cut vertices are still stored as the detail information $d_i$, why? Because when we apply a vertex split operation $vsplit_i$ to $M^i$, the connectivity of the split vertex $v_{s_i}$ and its 1-ring neighbors $N_{v_{s_{i}}}$ in the resulting mesh $M^{i+1}$ is determined by the \emph{cut vertices} of $vsplit_i$, which are originally stored in the detail information $d_i$.  \\

Furthermore, sequential LOD meshes $M^i$ can be naturally generated by using $vsplit_i$ and $ecol_i$ operations. The mesh of LOD lever $i$ can be produced bi-directionally by applying the PM sequence $(vsplit_0,\ldots,vsplit_{i-1})$ or $(ecol_{n-1},\ldots,ecol_{i})$ on the base mesh $M^0$ or on the original mesh $M^n$ respectively. (see \FG{fig:PmExample})
$$
	M^0\underoverrightleftarrow{ecol_{i-1}}{vsplit_{i-1}}\ldots\underoverrightleftarrow{ecol_{i-1}}{vsplit_{i-1}}M^i\underoverrightleftarrow{ecol_i}{vsplit_i}\ldots\underoverrightleftarrow{vsplit_{n-1}}{ecol_{n-1}}M^n
$$

\begin{figure}[htb]
	\centering
	
	\begin{pdfpic}
		\psline[linewidth=0.04cm](1.6570117,3.3332617)(2.5170116,3.1732616)
\psline[linewidth=0.04cm](2.7970116,2.8532617)(3.0570116,1.7932618)
\psline[linewidth=0.04cm](3.0370116,1.4132618)(2.3170118,0.4132617)
\psline[linewidth=0.04cm](0.97701174,0.7132617)(2.0170116,0.4132617)
\psline[linewidth=0.04cm](0.27701172,2.0132618)(0.6370117,1.0532618)
\psline[linewidth=0.04cm](0.73701173,3.0932617)(0.3170117,2.3332617)
\psline[linewidth=0.04cm](0.8970117,3.2532618)(1.4370117,3.3332617)
\psline[linewidth=0.04cm](1.5570117,3.1732616)(1.3970118,2.6532617)
\psline[linewidth=0.04cm](0.8970117,2.9732618)(1.2170117,2.6132617)
\psline[linewidth=0.04cm](1.6970117,2.5332618)(2.5570116,2.9732618)
\psline[linewidth=0.04cm](0.4770117,2.1532617)(1.1570117,2.3332617)
\psline[linewidth=0.04cm,linestyle=dotted,dotsep=0.16cm](1.5570117,2.1932616)(1.7970117,1.6132617)
\psline[linewidth=0.04cm](1.3370117,2.1932616)(0.91701174,1.1332617)
\psline[linewidth=0.04cm](1.7370117,2.3532617)(2.9370117,1.6532617)
\psline[linewidth=0.04cm](2.0770118,1.4932617)(2.8370118,1.4932617)
\psline[linewidth=0.04cm](1.9570117,1.1932617)(2.1570117,0.6132617)
\psline[linewidth=0.04cm](1.0370117,0.9332617)(1.6570117,1.3332617)
\usefont{T1}{ptm}{m}{n}
\rput(1.5491016,3.5382617){$v_0$}
\usefont{T1}{ptm}{m}{n}
\rput(2.7335448,3.1182618){$v_1$}
\usefont{T1}{ptm}{m}{n}
\rput(3.2480469,1.5982617){$v_2$}
\usefont{T1}{ptm}{m}{n}
\rput(2.2205763,0.3182617){$v_3$}
\usefont{T1}{ptm}{m}{n}
\rput(0.74927735,0.8982617){$v_4$}
\usefont{T1}{ptm}{m}{n}
\rput(0.16365235,2.1982617){$v_5$}
\usefont{T1}{ptm}{m}{n}
\rput(0.7485742,3.2982616){$v_6$}
\usefont{T1}{ptm}{m}{n}
\rput(1.4275196,2.4182618){$v_7$}
\usefont{T1}{ptm}{m}{n}
\rput(1.8451465,1.3982617){$v_8$}
\psline[linewidth=0.04cm](5.217012,3.3132617)(6.0770116,3.1532617)
\psline[linewidth=0.04cm](6.357012,2.8332617)(6.6170115,1.7732617)
\psline[linewidth=0.04cm](6.5970116,1.3932617)(5.877012,0.39326173)
\psline[linewidth=0.04cm](4.5370116,0.69326174)(5.5770116,0.39326173)
\psline[linewidth=0.04cm](3.8370118,1.9932617)(4.197012,1.0332618)
\psline[linewidth=0.04cm,linestyle=dotted,dotsep=0.16cm](4.297012,3.0732617)(3.8770118,2.3132617)
\psline[linewidth=0.04cm](4.4570117,3.2332618)(4.9970117,3.3132617)
\psline[linewidth=0.04cm](5.1170115,3.1532617)(5.1170115,2.3332617)
\psline[linewidth=0.04cm](4.4570117,2.9532616)(4.9370117,2.3732617)
\psline[linewidth=0.04cm](5.4170117,2.2132616)(6.0170116,2.8732617)
\psline[linewidth=0.04cm](4.0370116,2.1332617)(4.797012,2.0532618)
\psline[linewidth=0.04cm](5.0170116,1.6932617)(4.4770117,1.1132617)
\psline[linewidth=0.04cm](5.4570117,1.9132618)(6.4770117,1.6332617)
\psline[linewidth=0.04cm](5.277012,1.7332617)(5.717012,0.5932617)
\usefont{T1}{ptm}{m}{n}
\rput(5.109102,3.5182617){$v_0$}
\usefont{T1}{ptm}{m}{n}
\rput(6.293545,3.0982618){$v_1$}
\usefont{T1}{ptm}{m}{n}
\rput(6.808047,1.5782617){$v_2$}
\usefont{T1}{ptm}{m}{n}
\rput(5.780576,0.29826173){$v_3$}
\usefont{T1}{ptm}{m}{n}
\rput(4.3092775,0.87826174){$v_4$}
\usefont{T1}{ptm}{m}{n}
\rput(3.7236524,2.1782618){$v_5$}
\usefont{T1}{ptm}{m}{n}
\rput(4.308574,3.2782617){$v_6$}
\usefont{T1}{ptm}{m}{n}
\rput(5.1475196,1.9982617){$v_7$}
\psline[linewidth=0.04cm](9.937012,2.8132617)(10.197012,1.7532617)
\psline[linewidth=0.04cm](10.1770115,1.3732617)(9.457012,0.37326172)
\psline[linewidth=0.04cm,linestyle=dotted,dotsep=0.16cm](8.117012,0.6732617)(9.157012,0.37326172)
\psline[linewidth=0.04cm](7.5570116,2.2732618)(7.777012,1.0132617)
\psline[linewidth=0.04cm](9.077012,2.4732618)(9.597012,2.8532617)
\psline[linewidth=0.04cm](7.797012,2.4932618)(8.477012,2.2732618)
\psline[linewidth=0.04cm](8.637012,2.0732617)(8.057012,1.0932617)
\psline[linewidth=0.04cm](9.077012,2.2132616)(10.057012,1.6132617)
\psline[linewidth=0.04cm](8.897012,2.0132618)(9.297011,0.57326174)
\usefont{T1}{ptm}{m}{n}
\rput(9.873545,3.0782616){$v_1$}
\usefont{T1}{ptm}{m}{n}
\rput(10.268047,1.5182617){$v_2$}
\usefont{T1}{ptm}{m}{n}
\rput(9.360577,0.27826172){$v_3$}
\usefont{T1}{ptm}{m}{n}
\rput(7.8892775,0.8582617){$v_4$}
\usefont{T1}{ptm}{m}{n}
\rput(7.5236526,2.6582618){$v_5$}
\usefont{T1}{ptm}{m}{n}
\rput(8.7475195,2.3382616){$v_7$}
\psline[linewidth=0.04cm,arrowsize=0.05291667cm 2.0,arrowlength=1.4,arrowinset=0.4]{->}(6.5570116,3.7332618)(7.5370116,3.7332618)
\psline[linewidth=0.04cm,arrowsize=0.05291667cm 2.0,arrowlength=1.4,arrowinset=0.4]{->}(10.357012,0.6532617)(10.357012,-0.8267383)
\psline[linewidth=0.04cm,arrowsize=0.05291667cm 2.0,arrowlength=1.4,arrowinset=0.4]{->}(6.797012,-1.1467383)(7.777012,-1.1467383)
\psline[linewidth=0.04cm,arrowsize=0.05291667cm 2.0,arrowlength=1.4,arrowinset=0.4]{->}(3.6570117,-3.6267383)(2.6370118,-3.6267383)
\psline[linewidth=0.04cm,arrowsize=0.05291667cm 2.0,arrowlength=1.4,arrowinset=0.4]{<-}(6.9370117,-3.9667382)(7.9170117,-3.9667382)
\psline[linewidth=0.04cm,arrowsize=0.05291667cm 2.0,arrowlength=1.4,arrowinset=0.4]{<-}(3.8770118,-1.2467383)(2.5770118,-1.2467383)
\psline[linewidth=0.04cm,arrowsize=0.05291667cm 2.0,arrowlength=1.4,arrowinset=0.4]{<-}(8.457012,0.15326172)(8.457012,-0.6467383)
\psline[linewidth=0.04cm,arrowsize=0.05291667cm 2.0,arrowlength=1.4,arrowinset=0.4]{<-}(6.5170116,0.15326172)(7.4970117,0.15326172)
\psline[linewidth=0.04cm,arrowsize=0.05291667cm 2.0,arrowlength=1.4,arrowinset=0.4]{<-}(2.8370118,0.13326173)(3.8170118,0.13326173)
\psline[linewidth=0.04cm,arrowsize=0.05291667cm 2.0,arrowlength=1.4,arrowinset=0.4]{->}(2.8770118,3.7332618)(3.8570118,3.7332618)
\usefont{T1}{ptm}{m}{n}
\rput(3.318467,3.9782617){$ecol_4$}
\usefont{T1}{ptm}{m}{n}
\rput(6.998467,3.9782617){$ecol_3$}
\usefont{T1}{ptm}{m}{n}
\rput(11.038466,-0.44173828){$ecol_2$}
\usefont{T1}{ptm}{m}{n}
\rput(7.4184666,-3.7017384){$ecol_1$}
\usefont{T1}{ptm}{m}{n}
\rput(3.1384668,-3.9017382){$ecol_0$}
\usefont{T1}{ptm}{m}{n}
\rput(3.3584669,-0.14173828){$vsplit_4$}
\usefont{T1}{ptm}{m}{n}
\rput(7.078467,-0.10173828){$vsplit_3$}
\usefont{T1}{ptm}{m}{n}
\rput(9.298467,-0.28173828){$vsplit_2$}
\usefont{T1}{ptm}{m}{n}
\rput(7.178467,-0.8417383){$vsplit_1$}
\usefont{T1}{ptm}{m}{n}
\rput(3.1384668,-0.92173827){$vsplit_0$}
\psline[linewidth=0.04cm](8.957012,3.5532618)(9.557012,3.2132616)
\psline[linewidth=0.04cm](8.6770115,3.3932617)(8.717011,2.6532617)
\usefont{T1}{ptm}{m}{n}
\rput(8.689101,3.6582618){v0}
\psline[linewidth=0.04cm](7.877012,2.8532617)(8.437012,3.5332618)
\psline[linewidth=0.04cm](10.557012,-3.1067383)(9.537012,-3.7867384)
\psline[linewidth=0.04cm](8.097012,-2.2267382)(8.997012,-3.7667382)
\psline[linewidth=0.04cm](9.437012,-2.0067382)(9.977012,-1.6267383)
\psline[linewidth=0.04cm](8.1770115,-1.9867383)(8.797011,-2.1067383)
\psline[linewidth=0.04cm](9.437012,-2.3467383)(10.437012,-2.8667383)
\psline[linewidth=0.04cm,linestyle=dotted,dotsep=0.16cm](9.1770115,-2.4067383)(9.277012,-3.6667383)
\usefont{T1}{ptm}{m}{n}
\rput(10.253545,-1.4017383){$v_1$}
\usefont{T1}{ptm}{m}{n}
\rput(10.6480465,-2.9617383){$v_2$}
\usefont{T1}{ptm}{m}{n}
\rput(7.903652,-1.8217382){$v_5$}
\usefont{T1}{ptm}{m}{n}
\rput(9.12752,-2.1417382){$v_7$}
\psline[linewidth=0.04cm](9.337011,-0.92673826)(9.937012,-1.2667383)
\psline[linewidth=0.04cm](9.057012,-1.0867382)(9.117012,-1.8267382)
\usefont{T1}{ptm}{m}{n}
\rput(9.069101,-0.8217383){$v_0$}
\psline[linewidth=0.04cm](8.257011,-1.6267383)(8.817012,-0.9467383)
\psline[linewidth=0.04cm](10.337011,-1.6667383)(10.597012,-2.7267382)
\usefont{T1}{ptm}{m}{n}
\rput(9.189278,-4.021738){$v_4$}
\psline[linewidth=0.04cm](5.217012,-2.4467382)(5.777012,-2.1667383)
\psline[linewidth=0.04cm,linestyle=dotted,dotsep=0.16cm](3.9770117,-2.5267382)(4.5570116,-2.5467384)
\psline[linewidth=0.04cm](5.237012,-2.7267382)(6.237012,-3.4067383)
\usefont{T1}{ptm}{m}{n}
\rput(6.4480467,-3.5017383){$v_2$}
\usefont{T1}{ptm}{m}{n}
\rput(3.7036524,-2.3617382){$v_5$}
\usefont{T1}{ptm}{m}{n}
\rput(4.9275193,-2.6817384){$v_7$}
\psline[linewidth=0.04cm](5.0970116,-1.4467382)(5.697012,-1.7867383)
\psline[linewidth=0.04cm](4.857012,-1.6267383)(4.857012,-2.4467382)
\usefont{T1}{ptm}{m}{n}
\rput(4.8691015,-1.3617383){$v_0$}
\psline[linewidth=0.04cm](4.0570116,-2.1667383)(4.6170115,-1.4867383)
\psline[linewidth=0.04cm](6.1570115,-2.2067382)(6.4170117,-3.2667382)
\usefont{T1}{ptm}{m}{n}
\rput(5.993545,-1.9817383){$v_1$}
\psline[linewidth=0.04cm](1.8170117,-2.1667383)(2.3770118,-1.8867383)
\psline[linewidth=0.04cm](1.8370117,-2.4467382)(2.8370118,-3.1267383)
\usefont{T1}{ptm}{m}{n}
\rput(3.0480468,-3.2217383){$v_2$}
\usefont{T1}{ptm}{m}{n}
\rput(1.5275196,-2.4017382){$v_7$}
\psline[linewidth=0.04cm](1.6970117,-1.2067382)(2.2970116,-1.5467383)
\psline[linewidth=0.04cm](1.4570117,-1.3467383)(1.4570117,-2.1667383)
\usefont{T1}{ptm}{m}{n}
\rput(1.4691015,-1.1217383){$v_0$}
\psline[linewidth=0.04cm](2.7570117,-1.9267383)(3.0170116,-2.9867382)
\usefont{T1}{ptm}{m}{n}
\rput(2.593545,-1.7017382){$v_1$}
	\end{pdfpic} 
	\caption[Example of Progressive Mesh Sequence]{Example of PM sequence. Dashed edges are selected sequentially and collapse writ $ecol$ operation. And reversely, for each $M^i$, a $vsplit_i$ operation can be applied and reconstruct a new mesh $M^{i+1}$ with more details.}
	\label{fig:PmExample}

\end{figure}

\subsection{Quadric Error Metrics}
\label{subsection:theoreticalQEM}
\TODO{theoretical concepts of QEm}


\subsection{Vertices Hierarchy Forest}
\label{subsection:theoreticalVHF}
\TODO{theoretical concepts of VHF}


\section{Technical Concepts}
\label{section:TechConcpt}
\TODO{In this section, we will introduce the techniques we used in this project. }


%Picture
%\noindent
%\begin{minipage}{\linewidth}
%\makebox[\linewidth]{%
%\includegraphics[width=1.0\textwidth]{images/morphable.pdf}}
%\captionof{figure}{MorphableUI generates user-tailored interfaces for arbitrary applications in arbitrary environments. Users are able to use all available devices to control as many applications as needed. User behavior is analyzed by the system to increase the user experience.}% only if needed
%\label{fig:morphable}
%\bigskip
%\end{minipage}



\chapter{System Design}
\label{chapter:SystemDesign}
%\TODO{In this part we will introduce the software design of both server and client side. }

\begin{figure}[htb]
	\centering
	
	\includegraphics[scale=0.2]{graphs/images/client_server_illu.pdf}
	\caption{Application's Client and Server Architecture}
	\label{fig:clientserver}

\end{figure}
The main contribution of this thesis the view-dependent geometry streaming application implemented on mobile device. As stated in the previous chapters, our streaming application can be divided into to parts: client and server. (see \FG{fig:clientserver}) The client application is developed in Objective-C and deployed on a the iOS system. And the server part of the application is running on a an ordinary PC. In this chapter, software design detail of the both server-side and client-side will be described. \\

This chapter is organized as follows: First in \SC{section:serverdesign} we describe the software design of our geometry streaming server. We will focus on the software architecture of the server application. Then in \SC{section:clientdesign} we describe the software design of our geometry streaming client on iOS. And we will illustrate the implementation details in Chapter~\ref{chapter:SystemImplementation}.

\section{Server-side Design}
\label{section:serverdesign}

\begin{figure}[htb]
	\centering
	
	\begin{pdfpic}
	\psset{unit=0.90cm}
	\begin{pspicture}
	\psframe[linewidth=0.1,framearc=0.05,dimen=outer,shadow=true,shadowangle=-45.0,fillstyle=solid](16.0,4.4)(0.0,-4.4)
\psline[linewidth=0.2cm,arrowsize=0.073cm 2.6,arrowlength=0.6,arrowinset=0.3]{<-}(13.8,2.4)(13.8,-1.6)
\psline[linewidth=0.2cm,arrowsize=0.073cm 2.0,arrowlength=0.6,arrowinset=0.3]{<-}(6.4,2.4)(6.4,-1.6)
\psframe[linewidth=0.1,framearc=0.1,dimen=outer,shadow=true,shadowangle=-45.0,fillstyle=solid](7.8,1.8)(0.6,-0.8)
\psframe[linewidth=0.06,framearc=0.1,dimen=outer,shadow=true,shadowangle=-45.0,fillstyle=solid](3.0,1.0)(1.0,-0.4)
\psframe[linewidth=0.06,framearc=0.1,dimen=outer,shadow=true,shadowangle=-45.0,fillstyle=solid](5.2,1.0)(3.2,-0.4)
\psframe[linewidth=0.06,framearc=0.1,dimen=outer,shadow=true,shadowangle=-45.0,fillstyle=solid](7.4,1.0)(5.4,-0.4)
\usefont{T1}{ppl}{m}{n}
\rput(4.2684035,1.32){\small Network}
\usefont{T1}{ppl}{m}{n}
\rput(1.8684033,0.52){\small Network}
\usefont{T1}{ppl}{m}{n}
\rput(1.8211719,0.12){\small Protocol}
\usefont{T1}{ppl}{m}{n}
\rput(3.8788085,0.72){\small Socket}
\usefont{T1}{ppl}{m}{n}
\rput(4.140884,0.32){\small Transmitt}
\usefont{T1}{ppl}{m}{n}
\rput(4.1679397,-0.08){\small Controller}
\usefont{T1}{ppl}{m}{n}
\rput(6.042378,0.52){\small Vsplit}
\usefont{T1}{ppl}{m}{n}
\rput(6.356079,0.12){\small streaming}
\usefont{T1}{ppl}{m}{n}
\rput(7.8651757,3.605){\LARGE Progressive Mesh Server}
\psframe[linewidth=0.1,framearc=0.5,dimen=outer,shadow=true,shadowangle=-45.0,fillstyle=solid](15.8,3.2)(0.2,2.4)
\psframe[linewidth=0.1,framearc=0.1,dimen=outer,shadow=true,shadowangle=-45.0,fillstyle=solid](7.8,-1.4)(0.6,-3.8)
\psframe[linewidth=0.1,framearc=0.1,dimen=outer,shadow=true,shadowangle=-45.0,fillstyle=solid](15.4,1.8)(8.2,-0.8)
\psframe[linewidth=0.06,framearc=0.1,dimen=outer,shadow=true,shadowangle=-45.0,fillstyle=solid](4.0,-2.2)(1.0,-3.4)
\psframe[linewidth=0.06,framearc=0.1,dimen=outer,shadow=true,shadowangle=-45.0,fillstyle=solid](10.6,1.0)(8.6,-0.4)
\usefont{T1}{ppl}{m}{n}
\rput(4.04562,-1.88){\small Server Metadata}
\usefont{T1}{ppl}{m}{n}
\rput(11.9696,1.32){\small PM Manager}
\usefont{T1}{ppl}{m}{n}
\rput(2.4318457,-2.68){\small Server Info}
\psframe[linewidth=0.06,framearc=0.1,dimen=outer,shadow=true,shadowangle=-45.0,fillstyle=solid](12.8,1.0)(10.8,-0.4)
\psframe[linewidth=0.06,framearc=0.1,dimen=outer,shadow=true,shadowangle=-45.0,fillstyle=solid](15.0,1.0)(13.0,-0.4)
\usefont{T1}{ppl}{m}{n}
\rput(9.052651,0.52){\small PM}
\usefont{T1}{ppl}{m}{n}
\rput(9.6076565,0.12){\small Repository}
\usefont{T1}{ppl}{m}{n}
\rput(11.252651,0.52){\small PM}
\usefont{T1}{ppl}{m}{n}
\rput(11.5207615,0.12){\small Loader}
\usefont{T1}{ppl}{m}{n}
\rput(13.705142,0.52){\small VDPM}
\usefont{T1}{ppl}{m}{n}
\rput(13.798809,0.12){\small Support}
\psframe[linewidth=0.06,framearc=0.1,dimen=outer,shadow=true,shadowangle=-45.0,fillstyle=solid](7.4,-2.2)(4.4,-3.4)
\usefont{T1}{ppl}{m}{n}
\rput(5.790967,-2.68){\small Model List}
\psframe[linewidth=0.1,framearc=0.1,dimen=outer,shadow=true,shadowangle=-45.0,fillstyle=solid](15.4,-1.4)(8.2,-3.8)
\psframe[linewidth=0.06,framearc=0.1,dimen=outer,shadow=true,shadowangle=-45.0,fillstyle=solid](11.6,-2.2)(8.6,-3.4)
\usefont{T1}{ppl}{m}{n}
\rput(11.706909,-1.88){\small Server Rendering}
\usefont{T1}{ppl}{m}{n}
\rput(9.880346,-2.68){\small Renderer}
\psframe[linewidth=0.06,framearc=0.1,dimen=outer,shadow=true,shadowangle=-45.0,fillstyle=solid](15.0,-2.2)(12.0,-3.4)
\usefont{T1}{ppl}{m}{n}
\rput(13.061553,-2.68){\small Image}
\usefont{T1}{ppl}{m}{n}
\rput(13.497939,-3.08){\small Compressor}
\psline[linewidth=0.2cm,fillcolor=black,arrowsize=0.073cm 2.0,arrowlength=0.6,arrowinset=0.3]{<-}(2.4,2.4)(2.4,1.8)
\psline[linewidth=0.2cm,arrowsize=0.073cm 2.6,arrowlength=0.6,arrowinset=0.3]{<-}(10.2,2.4)(10.2,1.8)
\usefont{T1}{ppl}{m}{n}
\rput(8.085732,2.755){\large Server Central Controller}	
	\end{pspicture}
	\end{pdfpic} 
	\caption{Server Component Architecture}
	\label{fig:serverArch}

\end{figure}

\FG{fig:serverArch} shows the software components of the \emph{Progressive Mesh Server}. The server is consisting of four subsystems and one central controller. The four subsystems are (1) \emph{Network}, (2) \emph{PM Manager}, (3) \emph{Server Metadata} and (4) \emph{Server Rendering}. These four subsystems are responsible for their corresponding tasks and working collaboratively through the central controller. The rest part of this section describes in detail the design and function of each server subsystem separately. 


\subsection{Server PM Manager Subsystem}
\label{section:svrpmmangr}
\begin{figure}[htb]
	\centering
	
	\begin{pdfpic}
\psframe[linewidth=0.1,framearc=0.1,dimen=outer,shadow=true,shadowangle=-45.0,fillstyle=solid](8.0,2.6)(0.8,0.0)
\psframe[linewidth=0.06,framearc=0.1,dimen=outer,shadow=true,shadowangle=-45.0,fillstyle=solid](3.2,1.8)(1.2,0.4)
\usefont{T1}{ppl}{m}{n}
\rput(4.5695996,2.12){\small PM Manager}
\psframe[linewidth=0.06,framearc=0.1,dimen=outer,shadow=true,shadowangle=-45.0,fillstyle=solid](5.4,1.8)(3.4,0.4)
\psframe[linewidth=0.06,framearc=0.1,dimen=outer,shadow=true,shadowangle=-45.0,fillstyle=solid](7.6,1.8)(5.6,0.4)
\usefont{T1}{ppl}{m}{n}
\rput(1.6526513,1.32){\small PM}
\usefont{T1}{ppl}{m}{n}
\rput(2.2076561,0.92){\small Repository}
\usefont{T1}{ppl}{m}{n}
\rput(3.8526514,1.32){\small PM}
\usefont{T1}{ppl}{m}{n}
\rput(4.120762,0.92){\small Loader}
\usefont{T1}{ppl}{m}{n}
\rput(6.3051414,1.32){\small VDPM}
\usefont{T1}{ppl}{m}{n}
\rput(6.3988085,0.92){\small Support}
\psellipse[linewidth=0.04,dimen=outer](1.3,-1.1)(1.3,0.3)
\psbezier[linewidth=0.04](0.0,-1.8)(0.0,-2.2)(2.6,-2.2)(2.6,-1.8)
\psline[linewidth=0.04cm](0.0,-1.2)(0.0,-1.8)
\psline[linewidth=0.04cm](2.6,-1.2)(2.6,-1.8)
\usefont{T1}{ptm}{m}{n}
\rput(1.1862842,-1.695){\small Mesh Repo}
\psline[linewidth=0.08,arrowsize=0.05291667cm 2.0,arrowlength=1.4,arrowinset=0.4]{->}(2.2,0.4)(2.2,-0.6)(1.2,-0.6)(1.2,-1.2)
\psframe[linewidth=0.04,dimen=outer](4.8,-0.6)(2.8,-1.6)
\psframe[linewidth=0.04,dimen=outer,fillstyle=solid](5.0,-0.8)(3.0,-1.8)
\psframe[linewidth=0.04,dimen=outer,fillstyle=solid](5.2,-1.0)(3.2,-2.0)
\psline[linewidth=0.08,arrowsize=0.05291667cm 2.0,arrowlength=1.4,arrowinset=0.4]{->}(4.4,0.4)(4.4,-0.4)(3.8,-0.4)(3.8,-1.0)
\usefont{T1}{ptm}{m}{n}
\rput(4.0891895,-1.495){\small File Loading}
\psframe[linewidth=0.04,dimen=outer,fillstyle=solid](7.8,-0.8)(5.6,-2.0)
\psline[linewidth=0.08cm,arrowsize=0.05291667cm 2.0,arrowlength=1.4,arrowinset=0.4]{->}(6.6,0.4)(6.6,-0.8)
\psline[linewidth=0.05cm,linestyle=dashed,dash=0.16cm 0.16cm,arrowsize=0.05291667cm 2.0,arrowlength=1.4,arrowinset=0.4]{->}(5.6,-1.4)(5.2,-1.4)
\psline[linewidth=0.05,linestyle=dashed,dash=0.16cm 0.16cm,arrowsize=0.05291667cm 2.0,arrowlength=1.4,arrowinset=0.4]{->}(6.6,-2.0)(6.6,-2.6)(1.2,-2.6)(1.2,-2.1)
\usefont{T1}{ptm}{m}{n}
\rput(6.6638474,-1.495){File Parsing}
	\end{pdfpic} 
	\caption{PM Manager Subsystem}
	\label{fig:pmmanager}

\end{figure}
Server's PM Manager subsystem is mainly responsible for all the loading and parsing tasks of the progressive mesh models. Here we design the PM Manager subsystem as a utility component which can be accessed by other subsystems and the central controller as well. As is showed in \FG{fig:pmmanager} the \emph{PM Manager} subsystem is consisting of 3 main components - \emph{PM Repository}, \emph{PM Loader} and \emph{VDPM Support}. \\

First of all, the \emph{PM Manager} manages a progressive mesh repository which contains all the server side models as showed in \FG{fig:pmmanager}. Next, the \emph{PM Loader} is mainly responsible for all the I/O operations of the model files in the mesh repository. And then the VDPM Support part provide algorithm of parsing the loaded view-dependent progressive mesh files into corresponding data structure. We will discuss the details of parsing algorithm in Chapter~\ref{chapter:SystemImplementation}. 

\subsection{Server Network Subsystem}
\label{section:svrnetcom}
\begin{figure}[htb]
	\centering
	
	\begin{pdfpic}
	\begin{pspicture}
\psframe[linewidth=0.1,framearc=0.1,dimen=outer,shadow=true,shadowangle=-45.0,fillstyle=solid](7.2,1.3)(0.0,-1.3)
\psframe[linewidth=0.06,framearc=0.1,dimen=outer,shadow=true,shadowangle=-45.0,fillstyle=solid](2.4,0.5)(0.4,-0.9)
\psframe[linewidth=0.06,framearc=0.1,dimen=outer,shadow=true,shadowangle=-45.0,fillstyle=solid](4.6,0.5)(2.6,-0.9)
\psframe[linewidth=0.06,framearc=0.1,dimen=outer,shadow=true,shadowangle=-45.0,fillstyle=solid](6.8,0.5)(4.8,-0.9)
\usefont{T1}{ppl}{m}{n}
\rput(3.6684034,0.82){\small Network}
\usefont{T1}{ppl}{m}{n}
\rput(1.2684033,0.02){\small Network}
\usefont{T1}{ppl}{m}{n}
\rput(1.2211719,-0.38){\small Protocol}
\usefont{T1}{ppl}{m}{n}
\rput(3.2788086,0.22){\small Socket}
\usefont{T1}{ppl}{m}{n}
\rput(3.5408838,-0.18){\small Transmitt}
\usefont{T1}{ppl}{m}{n}
\rput(3.5679395,-0.58){\small Controller}
\usefont{T1}{ppl}{m}{n}
\rput(5.442378,0.02){\small Vsplit}
\usefont{T1}{ppl}{m}{n}
\rput(5.756079,-0.38){\small streaming}		
	\end{pspicture}
	\end{pdfpic} 
	\caption{Network Subsystem}
	\label{fig:netsubsys}

\end{figure}
The \emph{Network} subsystem can also be divided into three parts:
\begin{enumerate}
\item
Network Protocol
\item
Socket Transmit Controller
\item
Vsplit Streaming
\end{enumerate}
The Network Protocol part contains functionalities which handles the network communications and defines the protocol between the server and its corresponding client, including connection establishment, request/response handling, server-client synchronization, \etc.\\
The Socket transmit Controller part controls the socket functionalities. This part is relatively lower then the previous one. We design it to provide interfaces to other components for controlling the socket transmission operations, \ie listening on port, send bytes, \etc.\\
And the last part, Vsplit Streaming part provides all the functionalities of streaming vertex split information. Once the split information are ready, it will stream those data via the Socket Transmit Controller to the client side. 
\subsection{Server Metadata Subsystem}
\label{section:svrmtcom}
\begin{figure}[htb]
	\centering
	
	\begin{pdfpic}
	\psframe[linewidth=0.1,framearc=0.1,dimen=outer,shadow=true,shadowangle=-45.0,fillstyle=solid](7.2,1.2)(0.0,-1.2)
\psframe[linewidth=0.06,framearc=0.1,dimen=outer,shadow=true,shadowangle=-45.0,fillstyle=solid](3.4,0.4)(0.4,-0.8)
\usefont{T1}{ppl}{m}{n}
\rput(3.44562,0.72){\small Server Metadata}
\usefont{T1}{ppl}{m}{n}
\rput(1.8318458,-0.08){\small Server Info}
\psframe[linewidth=0.06,framearc=0.1,dimen=outer,shadow=true,shadowangle=-45.0,fillstyle=solid](6.8,0.4)(3.8,-0.8)
\usefont{T1}{ppl}{m}{n}
\rput(5.1909666,-0.08){\small Model List}
	\end{pdfpic} 
	\caption{Server Metadata Subsystem}
	\label{fig:metadata}

\end{figure}
\begin{figure}[htb]
	\centering
	
	\begin{pdfpic}
	\psframe[linewidth=0.1,framearc=0.05,dimen=outer,fillstyle=solid](4.88,2.64)(0.0,-2.64)
\usefont{T1}{ptm}{m}{n}
\rput(2.3972852,2.205){Model List}
\psframe[linewidth=0.04,framearc=0.3,dimen=outer,fillstyle=solid](4.12,1.92)(0.56,1.52)
\usefont{T1}{pcr}{m}{n}
\rput(1.544082,1.705){\small $Model_1$}
\rput{-90.0}(2.53,5.97){\pstriangle[linewidth=0.04,dimen=outer,fillstyle=solid](4.25,1.57)(0.36,0.3)}
\psframe[linewidth=0.04,framearc=0.3,dimen=outer,fillstyle=solid](4.12,1.44)(0.56,1.04)
\usefont{T1}{pcr}{m}{n}
\rput(1.544082,1.225){\small $Model_2$}
\rput{-90.0}(3.01,5.49){\pstriangle[linewidth=0.04,dimen=outer,fillstyle=solid](4.25,1.09)(0.36,0.3)}
\psframe[linewidth=0.04,framearc=0.3,dimen=outer,fillstyle=solid](4.1,0.94)(0.54,0.54)
\usefont{T1}{pcr}{m}{n}
\rput(1.5240821,0.725){\small $Model_3$}
\rput{-90.0}(3.49,4.97){\pstriangle[linewidth=0.04,dimen=outer,fillstyle=solid](4.23,0.59)(0.36,0.3)}
\psframe[linewidth=0.04,framearc=0.3,dimen=outer,fillstyle=solid](4.08,0.44)(0.52,0.04)
\usefont{T1}{pcr}{m}{n}
\rput(1.5040821,0.225){\small $Model_4$}
\rput{-90.0}(3.97,4.45){\pstriangle[linewidth=0.04,dimen=outer,fillstyle=solid](4.21,0.09)(0.36,0.3)}
\psframe[linewidth=0.04,framearc=0.3,dimen=outer,fillstyle=solid](4.08,-0.08)(0.52,-0.48)
\usefont{T1}{pcr}{m}{n}
\rput(1.5040821,-0.295){\small $Model_5$}
\rput{-90.0}(4.49,3.93){\pstriangle[linewidth=0.04,dimen=outer,fillstyle=solid](4.21,-0.43)(0.36,0.3)}
\psframe[linewidth=0.04,framearc=0.3,dimen=outer,fillstyle=solid](4.08,-1.74)(0.52,-2.14)
\usefont{T1}{pcr}{m}{n}
\rput(1.5040821,-1.955){\small $Model_n$}
\rput{-90.0}(6.15,2.27){\pstriangle[linewidth=0.04,dimen=outer,fillstyle=solid](4.21,-2.09)(0.36,0.3)}
\usefont{T1}{ptm}{m}{n}
\rput{-90.0}(3.3391309,1.3158692){\rput(2.2779396,-0.83){\huge ...}}
\psframe[linewidth=0.1,framearc=0.05,dimen=outer,fillstyle=solid](10.46,1.24)(7.26,-0.84)
\usefont{T1}{ptm}{m}{n}
\rput(8.627485,0.865){\small $ID_{model_i}$}
\usefont{T1}{ptm}{m}{n}
\rput(8.857486,0.465){\small $Name_{model_i}$}
\usefont{T1}{ptm}{m}{n}
\rput(8.747485,0.065){\small $Size_{model_i}$}
\usefont{T1}{ptm}{m}{n}
\rput(8.807486,-0.335){\small $Type_{model_i}$}
\psline[linewidth=0.04cm,arrowsize=0.05291667cm 2.0,arrowlength=1.4,arrowinset=0.4,doubleline=true,doublesep=0.12]{->}(5.06,0.16)(7.06,0.16)
\usefont{T1}{ptm}{m}{n}
\rput(5.831455,0.665){$Model_i$}
	\end{pdfpic} 
	\caption{Model List Illustration}
	\label{fig:modellist}

\end{figure}
Another part of the server architecture is the Server Metadata subsystem. It manages and maintains basic information of the server. As \FG{fig:metadata} shows, the metadata subsystem is consisted of two parts - \emph{Server Info} and \emph{Model List}. \emph{Server Info}. The \emph{Server Info} part contains network information of the server such as its IP address, port number, network bandwidth \etc. And The \emph{Model List} part provides a list view of all available model on the server. It also contains detail information of each model the server holds like model file name, file size, model type \etc. (See \FG{fig:modellist})

\subsection{Server Rendering Subsystem}
\label{section:svrrender}
\begin{figure}[htb]
	\centering
	
	\begin{pdfpic}
\psframe[linewidth=0.1,framearc=0.1,dimen=outer,shadow=true,shadowangle=-45.0,fillstyle=solid](11.939681,4.0)(4.739681,1.6)
\psframe[linewidth=0.06,framearc=0.1,dimen=outer,shadow=true,shadowangle=-45.0,fillstyle=solid](8.139681,3.2)(5.139681,2.0)
\usefont{T1}{ppl}{m}{n}
\rput(8.246591,3.52){\small Server Rendering}
\usefont{T1}{ppl}{m}{n}
\rput(6.4200277,2.72){\small Renderer}
\psframe[linewidth=0.06,framearc=0.1,dimen=outer,shadow=true,shadowangle=-45.0,fillstyle=solid](11.539681,3.2)(8.539681,2.0)
\usefont{T1}{ppl}{m}{n}
\rput(9.6012335,2.72){\small Image}
\usefont{T1}{ppl}{m}{n}
\rput(10.037621,2.32){\small Compressor}
\psframe[linewidth=0.06,framearc=0.1,dimen=outer,shadow=true,shadowangle=-45.0,fillstyle=solid](7.739681,1.0)(3.739681,0.0)
\psline[linewidth=0.06,arrowsize=0.05291667cm 2.0,arrowlength=1.4,arrowinset=0.4]{->}(6.539681,2.0)(6.539681,1.4)(6.139681,1.4)(6.139681,1.0)
\usefont{T1}{ptm}{m}{n}
\rput(5.822923,0.505){Refined Model}
\psframe[linewidth=0.06,framearc=0.1,dimen=outer,shadow=true,shadowangle=-45.0,fillstyle=solid](7.739681,-0.6)(3.739681,-1.6)
\usefont{T1}{ptm}{m}{n}
\rput(5.623626,-0.895){Geometry Submit}
\usefont{T1}{ptm}{m}{n}
\rput(5.4590073,-1.295){to Vertexbuffer}
\psline[linewidth=0.038cm,arrowsize=0.05291667cm 1.5,arrowlength=0.86,arrowinset=0.4,doubleline=true,doublesep=0.12]{->}(5.739681,0.0)(5.739681,-0.6)
\psframe[linewidth=0.06,framearc=0.1,dimen=outer,shadow=true,shadowangle=-45.0,fillstyle=solid](7.739681,-2.2)(3.739681,-3.2)
\usefont{T1}{ptm}{m}{n}
\rput(5.797689,-2.695){Render to Framebuffer}
\psline[linewidth=0.038cm,arrowsize=0.05291667cm 1.5,arrowlength=0.86,arrowinset=0.4,doubleline=true,doublesep=0.12]{->}(5.739681,-1.6)(5.739681,-2.2)
\psline[linewidth=0.06,arrowsize=0.05291667cm 2.0,arrowlength=1.4,arrowinset=0.4]{->}(9.939681,2.0)(9.939681,1.4)(10.539681,1.4)(10.539681,1.0)
\psframe[linewidth=0.06,framearc=0.1,dimen=outer,shadow=true,shadowangle=-45.0,fillstyle=solid](12.939681,1.0)(8.939681,0.0)
\usefont{T1}{ptm}{m}{n}
\rput(10.977972,0.705){Capture Bitmap Image}
\psframe[linewidth=0.06,framearc=0.1,dimen=outer,shadow=true,shadowangle=-45.0,fillstyle=solid](12.939681,-0.6)(8.939681,-1.6)
\usefont{T1}{ptm}{m}{n}
\rput(10.984407,-1.095){Image Compression}
\psline[linewidth=0.038cm,arrowsize=0.05291667cm 1.5,arrowlength=0.86,arrowinset=0.4,doubleline=true,doublesep=0.12]{->}(10.939681,0.0)(10.939681,-0.6)
\psframe[linewidth=0.06,framearc=0.1,dimen=outer,shadow=true,shadowangle=-45.0,fillstyle=solid](12.939681,-2.2)(8.939681,-3.2)
\usefont{T1}{ptm}{m}{n}
\rput(10.894144,-2.695){Image Ready}
\psline[linewidth=0.038cm,arrowsize=0.05291667cm 1.5,arrowlength=0.86,arrowinset=0.4,doubleline=true,doublesep=0.12]{->}(10.939681,-1.6)(10.939681,-2.2)
\psline[linewidth=0.04,arrowsize=0.05291667cm 1.5,arrowlength=0.86,arrowinset=0.4,doubleline=true,doublesep=0.12]{->}(5.739681,-3.2)(5.739681,-4.0)(8.459681,-4.0)(8.502181,0.4)(9.139681,0.4)
\usefont{T1}{ptm}{m}{n}
\rput(10.593822,0.305){from framebuffer}
\psframe[linewidth=0.04,linestyle=dashed,dash=0.16cm 0.16cm,framearc=0.2,dimen=outer](8.139681,1.2)(3.339681,-0.2)
\usefont{T1}{ptm}{m}{n}
\rput(2.4859114,2.105){\psframebox[linewidth=0.04]{PM Refinement Algorithm}}
\psbezier[linewidth=0.04,arrowsize=0.05291667cm 2.0,arrowlength=1.4,arrowinset=0.4]{->}(2.339681,1.8)(1.739681,0.8)(2.539681,0.6)(3.339681,0.4)
\psframe[linewidth=0.04,linestyle=dashed,dash=0.16cm 0.16cm,framearc=0.1,dimen=outer](8.139681,-0.4)(3.339681,-3.4)
\usefont{T1}{ptm}{m}{n}
\rput(1.4326888,-0.895){\psframebox[linewidth=0.04]{Done by OpenGL}}
\psbezier[linewidth=0.04,arrowsize=0.05291667cm 2.0,arrowlength=1.4,arrowinset=0.4]{->}(1.539681,-1.2)(1.539681,-2.0)(2.539681,-1.8857143)(3.339681,-2.0)
	\end{pdfpic} 
	\caption{Server Rendering}
	\label{fig:serverrendering}

\end{figure}
As is described in previous chapters, our application also has the functionality of server-side rendering when the client attempts to view a model with huge size. And the \emph{Server Rendering Subsystem} is mainly responsible for this task.\\ 

As is showed in \FG{fig:serverrendering}, the Server Rendering subsystem can be divided into two parts: \emph{Renderer} and \emph{Image Compressor}. \\

The Renderer part is responsible for all the graphics rendering tasks for server rendering. Generally the renderer is organized in a pipeline with three steps: (a) Model Refinement, (b) Geometry Submission and (c) Render to Framebuffer. As we can see from \FG{fig:serverrendering}, the first step is done by the progressive mesh refinement process and the next two steps is designed to be processed by server's GPU via OpenGL. \\

Next is the Image Compressor part. This part compresses the rendering result from GPU and get it ready for streaming. As is showed in \FG{fig:serverrendering}, it first captures the rendering result from GPU as bitmap image. Then the captured bitmap image is compressed using image compression algorithm. And Finally the image is prepared for streaming. 

\subsection{Server Central Controller}
\label{section:svrcencont}
\begin{figure}[htb]
	\centering
	
	\begin{pdfpic}
\psline[linewidth=0.2cm,arrowsize=0.073cm 2.6,arrowlength=0.6,arrowinset=0.3]{<-}(13.6,1.2)(13.6,-1.2)
\psline[linewidth=0.2cm,arrowsize=0.073cm 2.0,arrowlength=0.6,arrowinset=0.3]{<-}(6.2,1.2)(6.2,-1.0)
\psframe[linewidth=0.1,framearc=0.1,dimen=outer,shadow=true,shadowangle=-45.0,fillstyle=solid](7.6,0.6)(0.4,-0.4)
\usefont{T1}{ppl}{m}{n}
\rput(4.0684032,0.12){\small Network}
\psframe[linewidth=0.1,framearc=0.5,dimen=outer,shadow=true,shadowangle=-45.0,fillstyle=solid](15.6,2.0)(0.0,1.2)
\psframe[linewidth=0.1,framearc=0.1,dimen=outer,shadow=true,shadowangle=-45.0,fillstyle=solid](7.6,-1.0)(0.4,-2.0)
\psframe[linewidth=0.1,framearc=0.1,dimen=outer,shadow=true,shadowangle=-45.0,fillstyle=solid](15.2,0.6)(8.0,-0.4)
\usefont{T1}{ppl}{m}{n}
\rput(4.04562,-1.48){\small Server Metadata}
\usefont{T1}{ppl}{m}{n}
\rput(11.7696,0.12){\small PM Manager}
\psframe[linewidth=0.1,framearc=0.1,dimen=outer,shadow=true,shadowangle=-45.0,fillstyle=solid](15.2,-1.0)(8.0,-2.0)
\usefont{T1}{ppl}{m}{n}
\rput(11.706909,-1.48){\small Server Rendering}
\psline[linewidth=0.2cm,arrowsize=0.073cm 2.0,arrowlength=0.6,arrowinset=0.3]{<-}(2.2,1.2)(2.2,0.6)
\psline[linewidth=0.2cm,arrowsize=0.073cm 2.6,arrowlength=0.6,arrowinset=0.3]{<-}(10.0,1.2)(10.0,0.6)
\usefont{T1}{ppl}{m}{n}
\rput(7.8857327,1.555){\large Server Central Controller}
	\end{pdfpic} 
	\caption{Server Central Controller}
	\label{fig:servercentralcontroller}

\end{figure}

In the previous sections, we have described four main subsystems: Network, PM Manager, Server Metadata and Server Rendering. Each subsystem is responsible for their own tasks respectively. Obviously, a central management component is necessary for the four subsystems to collaborate together. The \emph{Server Central Controller} plays the role of centralized collaboration component.\\

As is illustrated in \FG{fig:servercentralcontroller}, The \emph{Server Central Controller} obtains connection with every subsystem. In other words, all the server behaviors are determined by the central controller. The central controller listens to client's request, determines vertex split details to send to client or employs the server rendering and captures image from frame buffer, \etc, during which each subsystem is invoked via the central controller. 



\section{Client-side Design}
\label{section:clientdesign}
\begin{figure}[htb]
	\centering
	
	\begin{pdfpic}
	\psset{unit=0.90cm}
	\begin{pspicture}
	\psframe[linewidth=0.1,framearc=0.05,dimen=outer,shadow=true,shadowangle=-45.0,fillstyle=solid](16.0,4.1)(0.0,-4.1)
\psline[linewidth=0.2cm,arrowsize=0.073cm 2.0,arrowlength=0.6,arrowinset=0.3]{<-}(6.4,2.3)(6.4,-1.7)
\psframe[linewidth=0.1,framearc=0.1,dimen=outer,shadow=true,shadowangle=-45.0,fillstyle=solid](7.8,1.7)(0.6,-1.1)
\psframe[linewidth=0.06,framearc=0.1,dimen=outer,shadow=true,shadowangle=-45.0,fillstyle=solid](3.0,0.9)(1.0,-0.7)
\psframe[linewidth=0.06,framearc=0.1,dimen=outer,shadow=true,shadowangle=-45.0,fillstyle=solid](5.2,0.9)(3.2,-0.7)
\psframe[linewidth=0.06,framearc=0.1,dimen=outer,shadow=true,shadowangle=-45.0,fillstyle=solid](7.4,0.9)(5.4,-0.7)
\usefont{T1}{ppl}{m}{n}
\rput(4.2684035,1.22){\small Network}
\usefont{T1}{ppl}{m}{n}
\rput(1.8684033,0.02){\small Network}
\usefont{T1}{ppl}{m}{n}
\rput(1.8211719,-0.38){\small Protocol}
\usefont{T1}{ppl}{m}{n}
\rput(3.8788085,0.42){\small Socket}
\usefont{T1}{ppl}{m}{n}
\rput(4.140884,0.02){\small Transmitt}
\usefont{T1}{ppl}{m}{n}
\rput(4.1679397,-0.38){\small Controller}
\usefont{T1}{ppl}{m}{n}
\rput(6.3387356,0.42){\small Receiving}
\usefont{T1}{ppl}{m}{n}
\rput(6.130947,-0.38){\small Stream}
\usefont{T1}{ppl}{m}{n}
\rput(8.159209,3.505){\LARGE iOS Client}
\psframe[linewidth=0.1,framearc=0.5,dimen=outer,shadow=true,shadowangle=-45.0,fillstyle=solid](15.8,3.1)(0.2,2.3)
\psframe[linewidth=0.1,framearc=0.1,dimen=outer,shadow=true,shadowangle=-45.0,fillstyle=solid](7.8,-1.5)(0.6,-3.7)
\psframe[linewidth=0.06,framearc=0.1,dimen=outer,shadow=true,shadowangle=-45.0,fillstyle=solid](4.0,-2.3)(1.0,-3.3)
\usefont{T1}{ppl}{m}{n}
\rput(4.40209,-1.98){\small View-Dependent Renderer}
\usefont{T1}{ppl}{m}{n}
\rput(2.514873,-2.78){\small OpenGL ES 2.0}
\psframe[linewidth=0.06,framearc=0.1,dimen=outer,shadow=true,shadowangle=-45.0,fillstyle=solid](7.4,-2.3)(4.4,-3.3)
\usefont{T1}{ppl}{m}{n}
\rput(5.60209,-2.58){\small Vertex Buffer}
\psline[linewidth=0.2cm,arrowsize=0.073cm 2.0,arrowlength=0.6,arrowinset=0.3]{<-}(2.4,2.3)(2.4,1.7)
\psline[linewidth=0.2cm,arrowsize=0.073cm 2.6,arrowlength=0.6,arrowinset=0.3]{<-}(10.2,2.3)(10.2,1.7)
\usefont{T1}{ppl}{m}{n}
\rput(8.034981,2.655){\large Client Central Controller}
\usefont{T1}{ppl}{m}{n}
\rput(1.6482276,0.42){\small Client}
\usefont{T1}{ppl}{m}{n}
\rput(6.042378,0.02){\small Vsplit}
\usefont{T1}{ppl}{m}{n}
\rput(5.0701365,-2.98){\small Object}
\psframe[linewidth=0.1,framearc=0.1,dimen=outer,shadow=true,shadowangle=-45.0,fillstyle=solid](15.4,-1.9)(8.2,-3.7)
\usefont{T1}{ppl}{m}{n}
\rput(11.731553,-2.18){\small Interactive User Interface}
\psframe[linewidth=0.06,framearc=0.1,dimen=outer,shadow=true,shadowangle=-45.0,fillstyle=solid](10.6,-2.5)(8.4,-3.3)
\usefont{T1}{ptm}{m}{n}
\rput(9.505332,-2.795){Model View}
\psframe[linewidth=0.06,framearc=0.1,dimen=outer,shadow=true,shadowangle=-45.0,fillstyle=solid](12.8,-2.5)(10.8,-3.3)
\usefont{T1}{ptm}{m}{n}
\rput(11.727285,-2.795){Mesh List}
\psframe[linewidth=0.06,framearc=0.1,dimen=outer,shadow=true,shadowangle=-45.0,fillstyle=solid](15.2,-2.5)(13.0,-3.3)
\usefont{T1}{ptm}{m}{n}
\rput(13.992598,-2.795){Setup}
\psline[linewidth=0.2cm,arrowsize=0.073cm 2.0,arrowlength=0.6,arrowinset=0.3]{<-}(13.2,2.3)(13.2,-1.9)
\psframe[linewidth=0.1,framearc=0.1,dimen=outer,shadow=true,shadowangle=-45.0,fillstyle=solid](15.4,1.7)(8.2,-1.7)
\psframe[linewidth=0.06,framearc=0.1,dimen=outer,shadow=true,shadowangle=-45.0,fillstyle=solid](15.0,0.9)(11.0,-0.1)
\usefont{T1}{ppl}{m}{n}
\rput(12.760307,0.42){\small Viewing Param Sync}
\psframe[linewidth=0.06,framearc=0.1,dimen=outer,shadow=true,shadowangle=-45.0,fillstyle=solid](15.0,-0.3)(8.6,-1.3)
\usefont{T1}{ppl}{m}{n}
\rput(11.784516,-0.78){\small View-dependent Refinement Algorithm}
\usefont{T1}{ppl}{m}{n}
\rput(11.9696,1.22){\small PM Manager}
\psframe[linewidth=0.06,framearc=0.1,dimen=outer,shadow=true,shadowangle=-45.0,fillstyle=solid](10.6,0.9)(8.6,-0.1)
\usefont{T1}{ppl}{m}{n}
\rput(9.584878,0.62){\small Base Mesh}
\usefont{T1}{ppl}{m}{n}
\rput(9.320762,0.22){\small Loader}	
	\end{pspicture}
	\end{pdfpic} 
	\caption{Client Component Architecture}
	\label{fig:clientArch}

\end{figure}
In \SC{section:serverdesign}, we describe the server-side component design of our application. Now in this section we will introduce the client-side design of our application. \\

First, let's take a look at \FG{fig:clientArch} which illustrates the component organization of the client. Similar to the server, our client system consists five collaborating components: (1) \emph{Client Network Component}, (2) \emph{Client PM Manager Component}, (3) \emph{Client View-dependent Renderer Component}, (4) \emph{Client Interactive User Interface Component} and the (5) \emph{Client Central Controller Component}. In the follow subsections, we will introduce these client-side components separately.

\subsection{Client Network Component}
\label{section:clientnetcom}
\begin{figure}[htb]
	\centering
	
	\begin{pdfpic}
	\psframe[linewidth=0.1,framearc=0.1,dimen=outer,shadow=true,shadowangle=-45.0,fillstyle=solid](9.332826,2.4756413)(2.1328256,-0.32435873)
\psframe[linewidth=0.06,framearc=0.1,dimen=outer,shadow=true,shadowangle=-45.0,fillstyle=solid](4.5328255,1.6756413)(2.5328255,0.075641274)
\psframe[linewidth=0.06,framearc=0.1,dimen=outer,shadow=true,shadowangle=-45.0,fillstyle=solid](6.7328258,1.6756413)(4.7328258,0.075641274)
\psframe[linewidth=0.06,framearc=0.1,dimen=outer,shadow=true,shadowangle=-45.0,fillstyle=solid](8.932825,1.6756413)(6.9328256,0.075641274)
\usefont{T1}{ppl}{m}{n}
\rput(5.801229,1.9956412){\small Network}
\usefont{T1}{ppl}{m}{n}
\rput(3.401229,0.7956413){\small Network}
\usefont{T1}{ppl}{m}{n}
\rput(3.3539975,0.39564127){\small Protocol}
\usefont{T1}{ppl}{m}{n}
\rput(5.411634,1.1956413){\small Socket}
\usefont{T1}{ppl}{m}{n}
\rput(5.6737094,0.7956413){\small Transmitt}
\usefont{T1}{ppl}{m}{n}
\rput(5.700765,0.39564127){\small Controller}
\usefont{T1}{ppl}{m}{n}
\rput(7.871561,1.1956413){\small Receiving}
\usefont{T1}{ppl}{m}{n}
\rput(7.6637726,0.39564127){\small Stream}
\usefont{T1}{ppl}{m}{n}
\rput(3.1810532,1.1956413){\small Client}
\usefont{T1}{ppl}{m}{n}
\rput(7.5752034,0.7956413){\small Vsplit}
\usefont{T1}{ptm}{m}{n}
\rput(2.6426399,-1.4193587){\psframebox[linewidth=0.04,linestyle=dashed,dash=0.16cm 0.16cm,framearc=0.1]{Client-side Transmisstion Protocol}}
\psline[linewidth=0.1,arrowsize=0.05291667cm 2.0,arrowlength=1.4,arrowinset=0.4]{->}(3.5328255,0.075641274)(3.5328255,-0.72435874)(2.5328255,-0.72435874)(2.5328255,-1.1243588)
\usefont{T1}{ptm}{m}{n}
\rput(4.75389,-2.2193587){\psframebox[linewidth=0.04,linestyle=dashed,dash=0.16cm 0.16cm,framearc=0.1]{Interact With Low-level Socket API}}
\psline[linewidth=0.1cm,arrowsize=0.05291667cm 2.0,arrowlength=1.4,arrowinset=0.4]{->}(5.7328258,0.075641274)(5.7328258,-1.9243587)
\usefont{T1}{ptm}{m}{n}
\rput(9.298763,-1.4193587){\psframebox[linewidth=0.04,linestyle=dashed,dash=0.16cm 0.16cm,framearc=0.1]{Listening on Socket & Receiving data}}
\psline[linewidth=0.1,arrowsize=0.05291667cm 2.0,arrowlength=1.4,arrowinset=0.4]{->}(7.9328256,0.075641274)(7.9328256,-0.72435874)(8.932825,-0.72435874)(8.932825,-1.1243588)	
	\end{pdfpic} 
	\caption{Client Network Component}
	\label{fig:clientnetwork}

\end{figure}
Similar to the Network subsystem in the server side, the \emph{Network Component} in the client side is consisted of three parts. 
\begin{enumerate}
\item
\emph{Client Network Protocol}\\
This part contains the network protocol for the client side. The client side protocol and server side protocol need to work together collaboratively to ensure communication and information synchronization between the server and client. 
\item
\emph{Socket Transmit Controller}\\
Just like the server side, the client uses socket for network communication. And this part performs communicate with low-level socket API for sending and receiving data through socket. 
\item
\emph{Receiving Vsplit Stream}\\
As is illustrated in \FG{fig:clientnetwork}, this part mainly responsible for receiving the vsplit data of the progressive mesh. It listens on socket and receive the detail information of a progressive mesh for refinement.
\end{enumerate}

\subsection{Client PM Manager Component}
\label{section:clientpmmcom}
\begin{figure}[htb]
	\centering
	
	\begin{pdfpic}
	\psframe[linewidth=0.1,framearc=0.1,dimen=outer,shadow=true,shadowangle=-45.0,fillstyle=solid](11.0,3.7)(3.8,0.3)
\psframe[linewidth=0.06,framearc=0.1,dimen=outer,shadow=true,shadowangle=-45.0,fillstyle=solid](10.6,2.9)(6.6,1.9)
\usefont{T1}{ppl}{m}{n}
\rput(8.360308,2.42){\small Viewing Param Sync}
\usefont{T1}{ppl}{m}{n}
\rput(7.5695996,3.22){\small PM Manager}
\psframe[linewidth=0.06,framearc=0.1,dimen=outer,shadow=true,shadowangle=-45.0,fillstyle=solid](10.6,1.7)(4.2,0.7)
\psframe[linewidth=0.06,framearc=0.1,dimen=outer,shadow=true,shadowangle=-45.0,fillstyle=solid](6.2,2.9)(4.2,1.9)
\usefont{T1}{ppl}{m}{n}
\rput(5.184878,2.62){\small Base Mesh}
\usefont{T1}{ppl}{m}{n}
\rput(4.9207616,2.22){\small Loader}
\psframe[linewidth=0.06,linestyle=dashed,dash=0.16cm 0.16cm,framearc=0.1,dimen=outer,fillstyle=solid](3.0,3.3)(0.0,1.5)
\usefont{T1}{ptm}{m}{n}
\rput(0.9611914,2.805){Load Base}
\usefont{T1}{ptm}{m}{n}
\rput(0.8077148,2.405){mesh on }
\usefont{T1}{ptm}{m}{n}
\rput(1.5047265,2.005){Initial Connection}
\psline[linewidth=0.1cm,arrowsize=0.05291667cm 2.0,arrowlength=1.4,arrowinset=0.4]{->}(4.2,2.5)(3.0,2.5)
\psframe[linewidth=0.06,linestyle=dashed,dash=0.16cm 0.16cm,framearc=0.1,dimen=outer,fillstyle=solid](11.0,0.1)(8.0,-1.7)
\usefont{T1}{ppl}{m}{n}
\rput(7.3845167,1.22){\small View-dependent Refinement Algorithm}
\psline[linewidth=0.1,arrowsize=0.05291667cm 2.0,arrowlength=1.4,arrowinset=0.4]{->}(10.6,2.5)(11.4,2.5)(11.4,-0.5)(11.0,-0.5)
\usefont{T1}{ptm}{m}{n}
\rput(9.540244,-0.195){Calculate Viewing}
\usefont{T1}{ptm}{m}{n}
\rput(9.396231,-0.595){Parameters from}
\psframe[linewidth=0.06,linestyle=dashed,dash=0.16cm 0.16cm,framearc=0.1,dimen=outer,fillstyle=solid](11.0,-2.3)(8.0,-3.7)
\usefont{T1}{ptm}{m}{n}
\rput(9.439834,-2.595){Sync with Server}
\usefont{T1}{ptm}{m}{n}
\rput(9.470879,-2.995){using Network}
\usefont{T1}{ptm}{m}{n}
\rput(9.433769,-3.395){Component}
\psline[linewidth=0.1cm,arrowsize=0.05291667cm 2.0,arrowlength=1.4,arrowinset=0.4]{->}(9.4,-1.7)(9.4,-2.3)
\usefont{T1}{ptm}{m}{n}
\rput(9.4207325,-0.995){Model and user's}
\usefont{T1}{ptm}{m}{n}
\rput(8.546826,-1.395){view}
\psline[linewidth=0.1,arrowsize=0.05291667cm 2.0,arrowlength=1.4,arrowinset=0.4]{->}(4.2,1.1)(3.0,1.1)(3.0,-0.3)(3.6,-0.3)
\psframe[linewidth=0.06,linestyle=dashed,dash=0.16cm 0.16cm,framearc=0.1,dimen=outer,fillstyle=solid](7.8,0.1)(3.6,-1.3)
\usefont{T1}{ptm}{m}{n}
\rput(5.566582,-0.195){Employ view-dependent}
\usefont{T1}{ptm}{m}{n}
\rput(5.4361424,-0.595){Refinement Algorithm}
\usefont{T1}{ptm}{m}{n}
\rput(5.7083983,-0.995){when receiving detail data}
	\end{pdfpic} 
	\caption{Client PM Manager Component}
	\label{fig:clientpmmanager}

\end{figure}
\emph{Progressive Mesh Manager} is the core component in the client side. As we can see in \FG{fig:clientpmmanager}, the PM Manager component includes (1) Base Mesh Loader, (2) Viewing Parameter Sync part and (3) View-dependent Refinement Algorithm part. These three parts together provide the functionality of the management and manipulation of Progressive Mesh in the client side. Now let's illustrate each part of the PM Manager Component. \\

The Base Mesh Loader is usually invoked at the very beginning of a model transmission. Its main responsibility is to load the base mesh transmitted from the server into client's memory.  Another part of of the \emph{Progressive Mesh Manager} is named Viewing Parameter Sync. It captures viewing parameters (incl. viewing angle, visible part, screen error... ) which assists the view-dependent refinement algorithm to determine which part of the model contributes to final rendered image. And then it uses network module to synchronize the viewing data with client's corresponding server. And the third part, View-dependent Refinement Algorithm part, is mainly responsible for employing view-dependent refinement algorithms to perform $ecol$ and $vsplit$ operations on current model. 


\subsection{Client View-dependent Renderer Component}
\label{section:clientvdrendercom}
\begin{figure}[htb]
	\centering
	
	\begin{pdfpic}
	\psframe[linewidth=0.1,framearc=0.1,dimen=outer,shadow=true,shadowangle=-45.0,fillstyle=solid](7.2,1.9)(0.0,-0.3)
\psframe[linewidth=0.06,framearc=0.1,dimen=outer,shadow=true,shadowangle=-45.0,fillstyle=solid](3.4,1.1)(0.4,0.1)
\usefont{T1}{ppl}{m}{n}
\rput(3.80209,1.42){\small View-Dependent Renderer}
\usefont{T1}{ppl}{m}{n}
\rput(1.914873,0.62){\small OpenGL ES 2.0}
\psframe[linewidth=0.06,framearc=0.1,dimen=outer,shadow=true,shadowangle=-45.0,fillstyle=solid](6.8,1.1)(3.8,0.1)
\usefont{T1}{ppl}{m}{n}
\rput(5.00209,0.82){\small Vertex Buffer}
\usefont{T1}{ppl}{m}{n}
\rput(4.4701366,0.42){\small Object}
\psframe[linewidth=0.06,linestyle=dashed,dash=0.16cm 0.16cm,framearc=0.1,dimen=outer,fillstyle=solid](3.4,-0.9)(0.4,-1.9)
\usefont{T1}{ptm}{m}{n}
\rput(1.8397559,-1.395){GLSL}
\psframe[linewidth=0.06,linestyle=dashed,dash=0.16cm 0.16cm,framearc=0.1,dimen=outer,fillstyle=solid](6.8,-0.9)(3.8,-1.9)
\usefont{T1}{ptm}{m}{n}
\rput(5.4318066,-1.595){Buffer Update}
\usefont{T1}{ptm}{m}{n}
\rput(5.3624415,-1.195){Continuously}
\psline[linewidth=0.04cm,arrowsize=0.05291667cm 2.0,arrowlength=1.4,arrowinset=0.4,doubleline=true,doublesep=0.12]{->}(2.0,0.1)(2.0,-0.9)
\psline[linewidth=0.04cm,arrowsize=0.05291667cm 2.0,arrowlength=1.4,arrowinset=0.4,doubleline=true,doublesep=0.12]{->}(5.4,0.1)(5.4,-0.9)

	\end{pdfpic} 
	\caption{Client View-dependent Renderer Component}
	\label{fig:clientvdrenderer}

\end{figure}

Similar to the server-side component, the client also has a renderer component. The view-dependent renderer's major task is to render the model on the screen of client device (iPad). We are using OpenGL ES 2.0 for the rendering job on the client side. For the purpose of code reuse, we employ the same piece of GLSL code both on the server- and client-side. As showed in \FG{fig:clientvdrenderer}, other part of the client-side renderer component is that we use the vertex buffer object on the client for storing model's geometry. Since the major feature of our application is streaming of mesh models, the model on the client-side is continuously changing. Therefore there will be numerous I/O operations on client's memory. For efficient memory update we create a component especially for manipulating the memory operations of the vertex buffer object. And in the situation of server rendering, the client renderer just simply display the rendered image transmitted from server. We will describe details of the implementation of this part in Chapter~\ref{chapter:SystemImplementation}. 


\subsection{Client Interactive User Interface Component}
\label{section:clientintuicom}
\input{graphs/GraphclientGUI}
One of the main contribution of our application is that it provides an \emph{Interactive User Interface} 
Another core component of the client is the \emph{Interactive User Interface} component. Here in this section we will describe in detail the design of client-side user interface. \\

As is showed in \FG{fig:clientgui}, there are three tabs in the \emph{Interactive User Interface}: (1) Model View, (2) Mesh List and (3) Setup. Users can switch freely through these three tabs via tab bar on the bottom of the screen.\\
\textbf{Model View} tab is the first tab of the GUI of our application. The Model View tab is designed in two perspectives: (a) \emph{visual effects} and (b) \emph{user interaction}. In other words, it is able to display the mesh model on the screen in response to users' touch gestures using the multi-touch feature of the device. This visual effects is the output of the view-dependent renderer mentioned in \SC{section:clientvdrendercom}. The client side renderer renders the mesh model with lighting effects and displays it on the screen. And meanwhile, users are able to rotate, move or zoom in/out the model with their finger gesture. We define the standard finger gestures as follows:
\begin{enumerate}
\item
Rotation: Use \textbf{one finger} to slide through the screen to rotate the viewing object. 
\item
Move: Use \textbf{two fingers} to slide parallely through the screen to move the viewing object.
\item
Zoom in/out: Use \textbf{two fingers}, glide them apart with continuous contact on the screen performs zoom in, and vice versa, glide them toward each other with continuous contact on the screen performs zoom out. 
\end{enumerate}
One important feature of the Model View tab is that its interactivity. Consider a scenario that a user rotates the model and zooms into a specific region of the model's surface. The system immediately starts to request for $vsplit$ details of the visible part of the currently model and continuously refines it. However the user may rotate or zoom again not until all the refinements are finished. In this situation, we design the visual effect of refinement process that it can always be interrupted by the user for the purpose of high interactivity. Implementation details of this part will be described in Chapter~\ref{chapter:SystemImplementation}.\\
\textbf{Mesh List} part is the second tab of our application user interface. As is introduced in \SC{section:svrmtcom}, the server-side maintains metadata, which contains a list of all available mesh models in server's mesh repository. Each time when a client process is connected to its server, the mesh list will be synchronized from server. And here in this Mesh List tab we display the list of all available meshes stored on the server. The user can select any item in the list to view it in the Model View tab. \\
\textbf{Setup} part is the last tab of our application user interface. This tab is used for configuration purpose. As is showed in \FG{fig:clientgui}, user can set the IP address, Port number of a mesh server. Furthermore, it is also possible to switch whether to render the model in server rendering mode or in client rendering mode. When client rendering mode is selected, the client-side application will continuously request for mesh details from its server. When server rendering mode is selected, instead of mesh details, the client-side application will continuously receive rendered images from its server.  

\subsection{Client Central Controller Component}
\label{section:clientcenconcom} 
\begin{figure}[htb]
	\centering
	
	\begin{pdfpic}
	\psset{unit=0.90cm}
	\begin{pspicture}
\psline[linewidth=0.2cm,arrowsize=0.073cm 2.0,arrowlength=0.6,arrowinset=0.3]{<-}(6.2,1.1)(6.2,-0.9)
\psframe[linewidth=0.1,framearc=0.1,dimen=outer,shadow=true,shadowangle=-45.0,fillstyle=solid](7.6,0.5)(0.4,-0.5)
\usefont{T1}{ppl}{m}{n}
\rput(4.0684032,0.02){\small Network}
\psframe[linewidth=0.1,framearc=0.5,dimen=outer,shadow=true,shadowangle=-45.0,fillstyle=solid](15.6,1.9)(0.0,1.1)
\psframe[linewidth=0.1,framearc=0.1,dimen=outer,shadow=true,shadowangle=-45.0,fillstyle=solid](7.6,-0.9)(0.4,-1.9)
\usefont{T1}{ppl}{m}{n}
\rput(4.20209,-1.38){\small View-Dependent Renderer}
\psline[linewidth=0.2cm,arrowsize=0.073cm 2.0,arrowlength=0.6,arrowinset=0.3]{<-}(2.2,1.1)(2.2,0.5)
\psline[linewidth=0.2cm,arrowsize=0.073cm 2.6,arrowlength=0.6,arrowinset=0.3]{<-}(10.0,1.1)(10.0,0.5)
\usefont{T1}{ppl}{m}{n}
\rput(7.8349805,1.455){\large Client Central Controller}
\psframe[linewidth=0.1,framearc=0.1,dimen=outer,shadow=true,shadowangle=-45.0,fillstyle=solid](15.2,-0.9)(8.0,-1.9)
\usefont{T1}{ppl}{m}{n}
\rput(11.731553,-1.38){\small Interactive User Interface}
\psline[linewidth=0.2cm,arrowsize=0.073cm 2.0,arrowlength=0.6,arrowinset=0.3]{<-}(13.0,1.1)(13.0,-0.9)
\psframe[linewidth=0.1,framearc=0.1,dimen=outer,shadow=true,shadowangle=-45.0,fillstyle=solid](15.2,0.5)(8.0,-0.5)
\usefont{T1}{ppl}{m}{n}
\rput(11.7696,0.02){\small PM Manager}
	\end{pspicture}
	\end{pdfpic} 
	\caption{Client Central Controller Component}
	\label{fig:clientcentral}

\end{figure}

Just like design of the server-side application, there is also a centralized component designed in the client to handle centralized management tasks. As is showed in \FG{fig:clientcentral}, the \emph{Central Controller} component has connection with all the other components. And any communication between different components will be handle by the Central Controller component. The implementation details of this part will also be described in Chapter~\ref{chapter:SystemImplementation}.



%Picture
%\noindent
%\begin{minipage}{\linewidth}
%\makebox[\linewidth]{%
%\includegraphics[width=1.0\textwidth]{images/morphable.pdf}}
%\captionof{figure}{MorphableUI generates user-tailored interfaces for arbitrary applications in arbitrary environments. Users are able to use all available devices to control as many applications as needed. User behavior is analyzed by the system to increase the user experience.}% only if needed
%\label{fig:morphable}
%\bigskip
%\end{minipage}



\chapter{System Implementation}
\label{chapter:SystemImplementation}
\TODO{In this chapter, we will describe in detail the implementation of both server and client of our mesh streaming framework.}

\section{Server Implementation}


\section{Client Implementation}

\section{Implementation Discussion}
\TODO{Discussion on implementation. }

%Picture
%\noindent
%\begin{minipage}{\linewidth}
%\makebox[\linewidth]{%
%\includegraphics[width=1.0\textwidth]{images/morphable.pdf}}
%\captionof{figure}{MorphableUI generates user-tailored interfaces for arbitrary applications in arbitrary environments. Users are able to use all available devices to control as many applications as needed. User behavior is analyzed by the system to increase the user experience.}% only if needed
%\label{fig:morphable}
%\bigskip
%\end{minipage}



\chapter{Evaluation}
\label{chapter:result}
%\TODO{In this chapter, we will describe the evaluation methodology and experimental results of our system. }
The main contribution of this thesis work is the view-dependent progressive mesh streaming framework on iPad. We have described and illustrated the design and implementation details of our system in previous chapters. In this chapter, we will do some experiment on our system and illustrate the evaluation results.  

\section{Overview}
\label{chapter:result:overview}
Since our framework provides both client and server rendering of progressive mesh, experiments are performed both for client rendering and server rendering scenario. The initial purpose of providing two ways of rendering is to process large models which are not suitable for client rendering, therefore different models will be used to test the system's performance in both scenarios. 
\subsection{Models for Test}
\label{chapter:result:overview:modelfortest}
Below is a list of test models: 
\begin{table}
\begin{center}
    \begin{tabular}{|	p{7.5cm}	|	l	|	l	|}
    \hline
    	\textbf{Model Name} 							& \textbf{Num. Vertices} 	& \textbf{Num. Faces}	\\ \hline
	bunny								&2503			&4968		\\ \hline
	bunny\_high							&34834			&69664		\\ \hline
	Armadillo\_mirror						&172974			&345944		\\ \hline
	hand									&210528			&416384		\\ \hline
	angel								&237018			&474048		\\ \hline
	hand\_new							&327323			&654666		\\ \hline
	dragon\_vrip							&437645			&871414		\\ \hline
    \end{tabular}
    \caption{Test models for client rendering.}
    \label{table:modelsclientrendering}
\end{center}
\end{table}

\begin{table}
\begin{center}
    \begin{tabular}{|	p{7.5cm}	|	l	|	l	|}
    \hline	
    	\textbf{Model Name} 						& \textbf{Num. Vertices} 	& \textbf{Num. Faces}	\\ \hline
    	happy\_vrip							&543652			&1087716		\\ \hline
	rolling\_stage\_1.2Mfaces\_edited			&596903			&1192501		\\ \hline
	810\_Red\_circular\_box\_1.4Mtriangles\_clean	&701332			&1402640		\\ \hline
	803\_neptune\_4Mtriangles\_manifold		&2003932			&4007872		\\ \hline
	Thai Statue							&5000000			&10000000	\\ \hline
    \end{tabular}
    \caption{Test models for server rendering.}
    \label{table:modelsserverrendering}
\end{center}
\end{table}

\TA{table:modelsclientrendering} lists test models for client rendering and \TA{table:modelsserverrendering} lists test models for server rendering. Models for both situations are sorted according to number of vertices and faces they contain. Here we decide to do server rendering for models with over 1M faces. 

\subsection{Hardware Setup}
%\TODO{This section describes hardwares we used for experiments. }\\
Since our framework is a client-server based application, we will illustration the hardware setup of both client and server side. \\

The \textbf{Client} side application is implemented and deployed on an \textbf{iPad 3rd generation\footnote{\label{ipad3rd}\url{http://support.apple.com/kb/SP647}}}. It has a Dual-core Apple A5X CPU (ARMv7) clocked at 1 GHz, with a system-on-chip quad-core graphics processor and 1 GB RAM. Its operating system is iOS 6.1.3. \\

And the \textbf{Server} side application is implemented and deployed on a \textbf{Macbook} with a Intel Core 2 Duo CPU of 2.4 GHz and 4 GB of DDR3 RAM. The server side operating system is MAC OS X 10.8.3. \\

The \textbf{Network Environment} we experiment in is a 10/100 M WiFi network environment with average roundtrip delay of 30.569 ms

\section{Experiment Results }
\label{section:expresult}
We evaluate our framework separately for server and client rendering scenario. And for each scenario, we measure our framework in three aspects: (1) Transmission Performance, (2) Memory/CPU Usage and (3) Visual Quality. \\

A standard testing process is defined. For each experiment, we will first open the model to test. (As default it will be put in the center of screen. ) Next we start the streaming process until maximum LOD is reached. And then we rotate to the back side and zoom-in to the center-part of the model. The test ends until maximum LOD is reached in current viewing situation. In the following sections we will illustrate the evaluation results of our experiments. 

\subsection{Client Rendering Evaluation}
\label{section:clienteva}
The client rendering evaluation experiments are performed on the models listed in \TA{table:modelsclientrendering}. 
\subsubsection{Transmission Performance}
\label{section:clienttransperf}
\begin{figure}[htb]
	\centering
	
	\begin{pdfpic}
\psline[linewidth=0.05cm,arrowsize=0.05291667cm 2.32,arrowlength=1.4,arrowinset=0.4]{->}(0.38910156,3.9910352)(0.38910156,-4.508965)
\usefont{T1}{ptm}{m}{n}
\rput(2.1228712,4.396035){Client}
\usefont{T1}{ptm}{m}{n}
\rput(5.5489354,4.396035){Server}
\psline[linewidth=0.02cm,linestyle=dashed,dash=0.16cm 0.16cm](2.0891016,4.1910353)(2.0891016,-4.508965)
\psline[linewidth=0.02cm,linestyle=dashed,dash=0.16cm 0.16cm](5.589102,4.1910353)(5.589102,-4.508965)
\usefont{T1}{ptm}{m}{n}
\rput{-90.0}(0.3924024,7.245801){\rput(3.7567968,3.651035){\Huge ...}}
\psline[linewidth=0.04cm,arrowsize=0.05291667cm 2.0,arrowlength=1.4,arrowinset=0.4]{->}(2.3891015,2.5910351)(5.2891016,2.0910351)
\usefont{T1}{ptm}{m}{n}
\rput{-8.649829}(-0.34278482,0.62180525){\rput(3.9170704,2.5960352){SyncViewingParam}}
\psline[linewidth=0.04cm,arrowsize=0.05291667cm 2.0,arrowlength=1.4,arrowinset=0.4]{<-}(2.5293014,0.5025208)(5.248902,1.2795495)
\usefont{T1}{ptm}{m}{n}
\rput{16.396927}(0.4885742,-1.0360752){\rput(3.8367383,1.1960351){VsplitStream}}
\psframe[linewidth=0.04,dimen=outer,fillstyle=solid](5.7891016,2.0910351)(5.3891015,1.1910352)
\usefont{T1}{ptm}{m}{n}
\rput(6.6628613,1.6960351){Processing}
\psframe[linewidth=0.04,dimen=outer,fillstyle=solid](2.2891016,0.29103515)(1.8891015,-0.60896486)
\usefont{T1}{ptm}{m}{n}
\rput(3.5271094,-0.10396484){Refine&Render}
\psframe[linewidth=0.04,dimen=outer,fillstyle=solid](2.2891016,-1.1089648)(1.8891015,-2.4089649)
\usefont{T1}{ptm}{m}{n}
\rput(3.4667382,-1.4039649){ViewingParam}
\psline[linewidth=0.05cm,arrowsize=0.05291667cm 2.32,arrowlength=1.4,arrowinset=0.4]{->}(2.0891016,-0.60896486)(2.0891016,-1.1089648)
\usefont{T1}{ptm}{m}{n}
\rput(3.0408204,-1.7039648){Decrease}
\psline[linewidth=0.04cm,arrowsize=0.05291667cm 2.0,arrowlength=1.4,arrowinset=0.4]{->}(2.2891016,-2.2089648)(5.4891014,-2.8089647)
\usefont{T1}{ptm}{m}{n}
\rput{-8.649829}(0.45317224,0.5464833){\rput(3.8170702,-2.703965){SyncViewingParam}}
\psframe[linewidth=0.04,dimen=outer,fillstyle=solid](5.7891016,-2.7089648)(5.3891015,-3.608965)
\usefont{T1}{ptm}{m}{n}
\rput{-90.0}(7.1924024,0.44580078){\rput(3.7567968,-3.148965){\Huge ...}}
\usefont{T1}{ptm}{m}{n}
\rput(0.35844725,4.396035){Time}
\usefont{T1}{ptm}{m}{n}
\rput(9.480206,1.4960351){Elapsed Time}
\psline[linewidth=0.04cm,arrowsize=0.05291667cm 2.0,arrowlength=1.4,arrowinset=0.4]{->}(7.7891016,0.99103516)(8.389102,1.3910352)
\usefont{T1}{ptm}{m}{n}
\rput{-90.0}(8.092402,-0.45419908){\rput(3.7567968,-4.048965){\Huge ...}}
\psframe[linewidth=0.04,dimen=outer,fillstyle=solid](2.2891016,3.5910351)(1.8891015,2.2910352)
\psframe[linewidth=0.05,linestyle=dashed,dash=0.16cm 0.16cm,framearc=0.25,dimen=outer](7.6891017,3.1910353)(1.3891015,-0.9089649)

	\end{pdfpic} 
	\caption{Client Rendering Transmission Elapsed Time Illustration}
	\label{fig:clientrndtransillu}

\end{figure}
\begin{figure}
\centering
\subfigure[b][Data Transmission. X Axis: time (second), Y Axis: Data (KB).]{
	\centering
	\includegraphics[width =\textwidth] {results/hand_trans_perf.pdf}
	\label{fig:hand_trans_perf_data}
}
\subfigure[b][Elapsed Time of each Viewing Parameter Synchronization. X Axis: Sync Request, Y Axis: Elapsed Time (second)]{
	\centering
	\includegraphics[width =\textwidth] {results/hand_trans_perf_elapsed_time.pdf}
	\label{fig:hand_trans_perf_elapsedtime}
}
\label{fig:hand_trans_perf}
\caption{Client Rendering Data Transmission Performance of Model "hand"}
\end{figure}

\begin{figure}
\centering
\subfigure[b][Data Transmission. X Axis: time (second), Y Axis: Data (KB).]{
	\centering
	\includegraphics[width =\textwidth] {results/dragon_vrip_trans_perf.pdf}
	\label{fig:dragon_vrip_trans_perf_data}
}
\subfigure[b][Elapsed Time of each Viewing Parameter Synchronization. X Axis: Sync Request, Y Axis: Elapsed Time (second)]{
	\centering
	\includegraphics[width =\textwidth] {results/dragon_vrip_trans_perf_elapsed_time.pdf}
	\label{fig:dragon_vrip_trans_perf_elapsedtime}
}
\label{fig:dragon_vrip_trans_perf}
\caption{Client Rendering Data Transmission Performance of Model "dragon\_vrip"}
\end{figure}


Recall the transmission process in the situation of client rendering. When a connection is established and there is no more user interaction on client's screen, the client application will repeatedly reduce the screen error tolerance by half and send the viewing  parameter with updated screen error tolerance to server for new $vsplit$ packets. Then $vsplit$ packets corresponding to current viewing parameter are transmitted to client and the client will perform refinement operation on current mesh. Once the refinement is finished, if there's no more user interaction,  the client will continue to decrease the screen error tolerance and perform the process again until current possible maximum LOD is reached. Therefore, we collect the elapsed time and $vsplit$ packet's size for each viewing parameter synchronization and response process. \FG{fig:clientrndtransillu} illustrates each sampled elapsed time interval. \\

\FGp{fig:hand_trans_perf_data} shows the data transmission performance experiment we performed on model "hand" (See \TAp{table:modelsclientrendering}). The X axis denotes the time passed and the Y axis denotes the data amount in the unit of KB. The blue curve illustrates the data transfer amount upon each viewing parameter synchronization request. And the red curve illustrates the accumulative data transfer amount over the time. It can be discovered that the major data transmission occurs between 0 - 80 seconds and 130 - 160 seconds. It is easy to explain: in the first 80 seconds we are streaming from the initial view of the model. And next crest reflects the rotation of view and the streaming process of model's back side geometry. During 80 - 130 seconds we are interacting with the model to adjust it to the second viewing angle and scale. From \FGp{fig:hand_trans_perf_data} we can also get the time of streaming from base mesh to highest LOD with maximum details for a specific viewing parameter with the model "hand". For the initial one it costs about 80 seconds. And for the second one it costs about 30 seconds. The reason why this time is decreasing is that in the second phase we are viewing (and zooming into) a small area of the model's back center and it naturally contains less detail to refine. \\

\FGp{fig:hand_trans_perf_elapsedtime} shows the elapsed time for each viewing parameter sync request of the experiment on model "hand". The elapsed time includes network transmission time and client-side refine\&render time. Naturally it has similar trend of the blue line in \FGp{fig:hand_trans_perf_data}.







\subsubsection{Memory/CPU Usage}
\label{section:clientmemcpuusage}

\begin{figure}
\centering
\subfigure[b][CPU Usage]{
	\centering
	\includegraphics[width =\textwidth] {results/hand_new_cpu.pdf}
	\label{fig:hand_new_cpu}
}
\subfigure[b][Memory Usage]{
	\centering
	\includegraphics[width =\textwidth] {results/hand_new_mem.pdf}
	\label{fig:hand_new_mem}
}
\label{fig:hand_new_cpu_mem}
\caption{CPU and Memory Usage of Experiment on model "hand\_new"}
\end{figure}
We use model "hand\_new" for our Memory/CPU Usage experiment. \FG{fig:hand_new_cpu} illustrates the CPU usage from the beginning of streaming. During the experiment the model is rotated and zoomed in for different viewing angle and scale. We can see from it that there are a climax of CPU usage between each idle time, which is the effects of each server processing between two viewing parameter synchronization request from server. \FG{fig:hand_new_mem} shows the memory usage. It shows that the server's memory consumption is quite steady during streaming. 


\subsubsection{Visual Quality}
\label{section:clientvisualquality}
\begin{figure}
\centering
\subfigure[b][]{
	\centering
	\includegraphics[width =0.45\textwidth] {results/030330.png}
	\label{fig:angel_visual_effects_1}
}
\hfill
\subfigure[b][]{
	\centering
	\includegraphics[width =0.45\textwidth] {results/030333.png}
	\label{fig:angel_visual_effects_2}
}

\subfigure[b][]{
	\centering
	\includegraphics[width =0.45\textwidth] {results/030337.png}
	\label{fig:angel_visual_effects_3}
}
\hfill
\subfigure[b][]{
	\centering
	\includegraphics[width =0.45\textwidth] {results/030348.png}
	\label{fig:angel_visual_effects_4}
}

\label{fig:angel_visual_effects}
\caption{Visual Effects of Model "angel"}

\end{figure}

\begin{figure}
\centering

\subfigure[b][]{
	\centering
	\includegraphics[width =0.45\textwidth] {results/030442.png}
	\label{fig:angel_visual_effects_hair2}
}
\hfill
\subfigure[b][]{
	\centering
	\includegraphics[width =0.45\textwidth] {results/030459.png}
	\label{fig:angel_visual_effects_hair4}
}

\subfigure[b][]{
	\centering
	\includegraphics[width =0.45\textwidth] {results/030519.png}
	\label{fig:angel_visual_effects_hair6}
}
\hfill
\subfigure[b][]{
	\centering
	\includegraphics[width =0.45\textwidth] {results/030607.png}
	\label{fig:angel_visual_effects_hair8}
}

\label{fig:angel_visual_effects_hair}
\caption{Visual Effects of Model "angel"'s "hair"}
\end{figure}
When the client side starts to stream a mesh, it will be viewed on the client screen and refined on-the-fly, which means it can be seen as a refinement animation. Therefore we choose to illustrate this process in a sequence of screen shots using the model "angel". 







\subsection{Server Rendering Evaluation}
\label{section:servereva}
\begin{figure}
\centering
\subfigure[b][Size of Data Transfer. For each viewing param sync request, an image will be rendered in corresponding LOD and transmitted to client. We can see that the time of server side refinement and rendering is quite stable and the accumulative size of data transmitted are linearly increasing, as expected.]{
	\centering
	\includegraphics[width =\textwidth] {results/hapy_vrip_trans_datasize.pdf}
	\label{fig:hapy_vrip_svr_data_trans_datasize}
}
\subfigure[b][Num. of active vertices and faces of the server-hold model during streaming. The low points in the middle of the graph indicates the user is changing view and unnecessary edges are collapsed, unnecessary vertices are eliminated.]{
	\centering
	\includegraphics[width =\textwidth] {results/hapy_vrip_vfront_over_time.pdf}
	\label{fig:hapy_vrip_svr_data_trans_active_faces_vertices}
}
\label{fig:hapy_vrip_svr_data_trans}
\caption{Server Rendering Data Transmission Statistics of model "hapy\_vrip"}
\end{figure}
\begin{figure}
\centering
\subfigure[b][Data Transmission. X Axis: time (second), Y Axis: Data (KB).]{
	\centering
	\includegraphics[height=0.3\textheight] {results/thai_trans_data_perf.pdf}
	\label{fig:thai_trans_data}
}
\subfigure[b][Number of vertices and faces of the model refined during streaming. X Axis: Vertices/Faces Number, Y Axis: Time (second)]{
	\centering
	\includegraphics[height=0.3\textheight] {results/thai_trans_geo_perf.pdf}
	\label{fig:thai_trans_geo}
}

\subfigure[b][Number of vertices and faces of the model refined during streaming compared with time elapsed.]{
	\centering
	\includegraphics[height=0.3\textheight] {results/thai_trans_vft.pdf}
	\label{fig:thai_trans_vft}
}
\label{fig:thai_trans_perf}
\caption{Experiment results of model Thai Statue of 10 million triangles.}
\end{figure}



The server rendering evaluation experiments are performed on the models listed in \TA{table:modelsserverrendering}.

\subsubsection{Transmission Performance}
\label{section:servertransperf}
Since in the server rendering situation, both the refinement and rendering is done by the server, the definition of elapsed time becomes: $Time_{refinement}+Time_{rendering}+Time_{transmission}$. We choose the model "hapy\_vrip" for experiment. Details see \FG{fig:hapy_vrip_svr_data_trans_datasize} and \FG{fig:hapy_vrip_svr_data_trans_active_faces_vertices}. We can see that the server rendering of big meshes is quite fast. 

Moreover, we also did some experiments on more powerful workstations with larger models. We did server rendering experiment on the model Thai Statue from Stanford Scanning Repository, which has 10M faces. As showed in \FG{fig:thai_trans_data}, the data transferred in each request is stable since the server is just streaming images. \FG{fig:thai_trans_geo} shows the geometry statistics during streaming. And \FG{fig:thai_trans_vft} shows the geometry statistics compared with elapsed time of each request. Hypothetically the elapsed time should has linear relationship with the number of vertices and faces refined. However, it is interesting to find that there are two high points of elapsed time in the chart and their corresponding number of faces and vertices are relatively low. The reason of this is that during the time of those two high points, user has interrupted the refinement and rendering and is changing his view. And meanwhile, the server is deleting unnecessary geometry of the model. 


\subsubsection{Visual Quality}
\label{section:servervisualquality}

\begin{figure}
\centering
\subfigure[b][Visual Effects at low LOD level]{
	\centering
	\includegraphics[width =0.3\textwidth] {results/170035.png}
	\label{fig:neptune_serverrendering_visual_effects1}
}
\hfill
\subfigure[b][Visual Effects at medium LOD level]{
	\centering
	\includegraphics[width =0.3\textwidth] {results/170038.png}
	\label{fig:neptune_serverrendering_visual_effects2}
}
\hfill
\subfigure[b][Visual Effects at high LOD level]{
	\centering
	\includegraphics[width =0.3\textwidth] {results/170041.png}
	\label{fig:neptune_serverrendering_visual_effects3}
}

\label{fig:neptune_serverrendering_visual_effects}
\caption{Server Rendering Visual Effects of model "neptune" with 4M triangles}
\end{figure}


\begin{figure}
\centering
\subfigure[b][Visual Effects at low LOD level]{
	\centering
	\includegraphics[width =0.3\textwidth] {results/Thai_1.png}
	\label{fig:thai_serverrendering_visual_effects1}
}
\hfill
\subfigure[b][Visual Effects at medium LOD level]{
	\centering
	\includegraphics[width =0.3\textwidth] {results/Thai_2.png}
	\label{fig:thai_serverrendering_visual_effects2}
}
\hfill
\subfigure[b][Visual Effects at high LOD level]{
	\centering
	\includegraphics[width =0.3\textwidth] {results/Thai_3.png}
	\label{fig:Thai_serverrendering_visual_effects3}
	}
\subfigure[b][Visual Effects at low LOD level]{
	\centering
	\includegraphics[width =0.3\textwidth] {results/Thai_4.png}
	\label{fig:thai_serverrendering_visual_effects4}
}
\hfill
\subfigure[b][Visual Effects at medium LOD level]{
	\centering
	\includegraphics[width =0.3\textwidth] {results/Thai_5.png}
	\label{fig:thai_serverrendering_visual_effects5}
}
\hfill
\subfigure[b][Visual Effects at high LOD level]{
	\centering
	\includegraphics[width =0.3\textwidth] {results/Thai_6.png}
	\label{fig:thai_serverrendering_visual_effects6}
}

\label{fig:thai_serverrendering_visual_effects}
\caption{Server Rendering Visual Effects of model "Thai Statue" with 10M triangles}
\end{figure}

Similar to \SC{section:clientvisualquality}, in this section we will illustrate some screen shots to show the visual quality of server rendering streaming. It can be found that there is significant different between \FG{fig:neptune_serverrendering_visual_effects1} and \FG{fig:neptune_serverrendering_visual_effects2}, while \FG{fig:neptune_serverrendering_visual_effects2} and \FG{fig:neptune_serverrendering_visual_effects3} looks almost the same. So as the visual effects showed in \FG{fig:thai_serverrendering_visual_effects1} to \FG{fig:thai_serverrendering_visual_effects6}This is because when model's LOD reaches a certain high level, the triangles reconstructed could even be smaller than a pixel on screen, thus making it looks no difference compared with lower LOD level. 

\section{Discussion}
\label{section:results:discussion}
%\TODO{Discussion of test results. }
In the previous paragraphs, we have illustrated the result of experiment on our framework. It can be found that for both client and server rendering scenario, our framework is able to provide stable functionality of view-dependent progressive mesh streaming with high visual quality and performance. \\

In the scenario of client rendering, the performance bottleneck is mainly the client side since the model will finally reconstructed and rendered by the client and meanwhile the client has to support interruptible user interaction during refinement and rendering. On the other hand, in the scenario of server rendering, performance burden on client is relatively light and the main performance bottleneck is in the server side. The server needs to do almost every thing including refinement and rendering. And the client just have to show the rendered images from server. \\ 


%Picture
%\noindent
%\begin{minipage}{\linewidth}
%\makebox[\linewidth]{%
%\includegraphics[width=1.0\textwidth]{images/morphable.pdf}}
%\captionof{figure}{MorphableUI generates user-tailored interfaces for arbitrary applications in arbitrary environments. Users are able to use all available devices to control as many applications as needed. User behavior is analyzed by the system to increase the user experience.}% only if needed
%\label{fig:morphable}
%\bigskip
%\end{minipage}



%\chapter{Collaboration with Other Projects}
\label{chapter:CollaborationWithOtherProjects}
\TODO{In this chapter we will introduce the collaboration with other projects. Especially with the Hybrid Rendering Server Project.}



%Picture
%\noindent
%\begin{minipage}{\linewidth}
%\makebox[\linewidth]{%
%\includegraphics[width=1.0\textwidth]{images/morphable.pdf}}
%\captionof{figure}{MorphableUI generates user-tailored interfaces for arbitrary applications in arbitrary environments. Users are able to use all available devices to control as many applications as needed. User behavior is analyzed by the system to increase the user experience.}% only if needed
%\label{fig:morphable}
%\bigskip
%\end{minipage}



\chapter{Conclusion}
\label{chapter:Conclusion}
\TODO{Conclusion of this project. Recap the contribution. }
\section{Future Work}
\TODO{Discuss potential future work. like discussion of Quadric Error metrics ... advantage and disadvantage}



%Picture
%\noindent
%\begin{minipage}{\linewidth}
%\makebox[\linewidth]{%
%\includegraphics[width=1.0\textwidth]{images/morphable.pdf}}
%\captionof{figure}{MorphableUI generates user-tailored interfaces for arbitrary applications in arbitrary environments. Users are able to use all available devices to control as many applications as needed. User behavior is analyzed by the system to increase the user experience.}% only if needed
%\label{fig:morphable}
%\bigskip
%\end{minipage}










%%%
%%% end main document
%%%
%%%%%%%%%%%%%%%%%%%%%%%%%%%%%%%%%%%%%%%%%%%%%%%%%%%%%%%%%%%%%%%%%%%%%%%%%%%%%%%%

\appendix  %% include it, if something (bibliography, index, ...) follows below

%%%%%%%%%%%%%%%%%%%%%%%%%%%%%%%%%%%%%%%%%%%%%%%%%%%%%%%%%%%%%%%%%%%%%%%%%%%%%%%%
%%%
%%% bibliography
%%%
%%% available styles: abbrv, acm, alpha, apalike, ieeetr, plain, siam, unsrt
%%%
\bibliographystyle{acm}
%\bibliographystyle{alpha}
%\bibliographystyle{apalike}

%%% name of the bibliography file without .bib
%%% e.g.: literatur.bib -> \bibliography{literatur}
\bibliography{thesis}
\end{spacing}
\end{document}
%%% }}}
%%% END OF FILE
%%%%%%%%%%%%%%%%%%%%%%%%%%%%%%%%%%%%%%%%%%%%%%%%%%%%%%%%%%%%%%%%%%%%%%%%%%%%%%%%
%%% Notice!
%%% This file uses the outline-mode of emacs and the foldmethod of Vim.
%%% Press 'zi' to unfold the file in Vim.
%%% See ':help folding' for more information.
%%%%%%%%%%%%%%%%%%%%%%%%%%%%%%%%%%%%%%%%%%%%%%%%%%%%%%%%%%%%%%%%%%%%%%%%%%%%%%%%
%% Local Variables:
%% mode: outline-minor
%% OPToutline-regexp: "%% .*"
%% OPTeval: (hide-body)
%% emerge-set-combine-versions-template: "%a\n%b\n"
%% End:
%% vim:foldmethod=marker

