\chapter{Introduction}
\label{chapter:introduction}
\TODO{Here is just a test for literature citation. \\
\cite{Krekhov:12:MSc}, \cite{Hoppe:1996:PM}, \cite{Hoppe:1997:VRP}, \cite{Kim:2001:trulyselective}, \cite{Kim:2003:TransitiveMeshSpace}, \cite{Kim:04:VDstreaming}, \cite{Yang:2004:VDMeshTrans}, \cite{Deb:2004:DesignStreamSys}
, \cite{Callahan:2006:PVR}, \cite{Pacanowski:2008:ESS}, \cite{Cheng:2007:AMP}, \cite{Bajaj:1999:PCTriMesh}, \cite{Khodakovsky:2000:PGC}, \cite{Deb:2006:RSRT}, \cite{Hoppe:1998:EIPM}
}
\\
\TODO{introduce this paper! }
\\

With the ever fast development of modern computer science, computer graphics and visualization has become a big topic. And with the more and more advanced 3D scanner and surface reconstruction technology, people are able to get extremely large 3D model with much more detail than ever before that are scanned from real objects such as sculptures. 
And mean while, in recent years, various mobile devices (such as iPhone, iPad, Google Nexus series, etc.) with much powerful computing resource are being designed and manufactured. With faster CPU/GPU and larger memory, these hand-held devices are made possible to run graphics programs or to view 3D models. These two trends have raised many topics about graphics development on mobile devices. The visualisation of large-scale 3D model is one of the hot topics among them. 

\smallskip

In this thesis, we proposed a streaming framework for large scale geometry models. In this framework users can connect our server from a mobile device (e.g. iPad) and view the 3D geometry model they choose progressively. User can browse the model using drag-and-zoom gesture on the multi-touch screen of the device and the system will refine the corresponding part of the viewing model according to users' viewing angle. Therefore our system can provide view-dependent, selective geometry streaming. 

\smallskip
In the following paragraphs of this chapter, we will introduce in detail the motivation of the thesis in Section~\ref{section:motivation}. And in Section~\ref{section:background} we will introduce the background of this topic. And Section~\ref{section:relWork} related works in the area of this topic will be listed and discussed. 

\smallskip
And then In Chapter~\ref{chapter:BasicConcepts}, we will describe some basic theoretical and technical concepts behind the topic of this thesis including \emph{Progressive Mesh}, the \emph{Quadric Error Metrics}, \emph{View-dependent Progressive Mesh}, \etc

\smallskip
In Chapter~\ref{chapter:SystemDesign}, we will describe the overall system architecture design including the server architecture and client architecture. 

\smallskip
In Chapter~\ref{chapter:SystemImplementation}, we will describe in detail about the implementation of our geometry streaming framework. In this chapter, both server and client side implementation will be illustrated. And after that there will be a short discussion of the implementation. 

\smallskip
In Chapter~\ref{chapter:ExperimentalEvaluation} and Chapter~\ref{chapter:CollaborationWithOtherProjects} we will show the experiment result of our framework and we will also describe the use of our framework in other projects. 

In the last chapter, we will conclude this thesis. Future work will also be discussed in this chapter. 

\section{Motivation}
\label{section:motivation}
%\TODO{In this section the motivation of our project will be described.}
In this section, we will describe the motivation of this thesis. 

\smallskip
Nowadays, with the fast development of hardware and computing power, more and more complex geometry models are being used for viewing more details and better visual quality. 

\smallskip
Maybe 10 years ago, people may thought the Stanford Bunny model with almost 70,000 triangles would already enough to show its surface details. But today, with more powerful hardwares, models with much more triangles are becoming increasingly popular. For instance, models we use from \textbf{The Stanford 3D Scanning Repository\footnote{\label{S3DSR}\url{http://graphics.stanford.edu/data/3Dscanrep/}}} are almost with over 10 million triangles. People using those large models in various application such as 3D game, industrial design etc.
 
\smallskip
In the traditional local application scenario, with a desktop PC with a powerful GPU, we can view this kind of large scale geometry models easily. However, problems came out when we want to view them via internet. In the traditional local application scenario, the whole model is downloaded once and then being rendered to screen. This approach is intuitive and efficient when dealing with a model which is relatively small. But when models over hundreds megabytes are being transmitted over the internet, long waiting time becomes a problem. Under such circumstance, we will lose rendering quality and users' satisfactory as well. 

\smallskip
And meanwhile, another development trend of today's information technology is that everything is going mobilised! From the first iPhone, to Android, from the release of Microsoft's Surface tablet to the reborn of Blackberry 10. We can see that almost suddenly those handheld devices have dominated our life. You can see people using an iPad, Nexus 7 or Windows 8 table everywhere. And this also inspire us to build our project's application over a handheld device: iPad. 

\smallskip
With handheld devices, new problems are raised. There are various limitations of those handheld devices such as less powerful computing resource, smaller size of RAM, and limited network connectivity. It's already stressful for a desktop to view a large model in the traditional download-whole-once scenario, not to mention how difficult it would be for a mobile device. 

\smallskip
Therefore, intuitively \emph{streaming approach} has been raised into our mind. That is the motivation of this project. In our approach, a base, coarsen geometry model will be transmitted to the client side upon an initial connections attempt. And after that, model details will be transmitted and the client-side model will be refined on the fly. Similar to video stream on \textbf{Youtube\footnote{\label{UTUBE}\url{http://www.youtube.com/}} }. And after the refinement streaming is finished, we can get the original mesh. Users can view the model during the refinement process. 


\section{Background}
\label{section:background}
\TODO{Here we will introduce background information of this topic.}

\section{Related Work}
\label{section:relWork}
\TODO{In this section we will introduce previous research in this area. }

%Picture
%\noindent
%\begin{minipage}{\linewidth}
%\makebox[\linewidth]{%
%\includegraphics[width=1.0\textwidth]{images/morphable.pdf}}
%\captionof{figure}{MorphableUI generates user-tailored interfaces for arbitrary applications in arbitrary environments. Users are able to use all available devices to control as many applications as needed. User behavior is analyzed by the system to increase the user experience.}% only if needed
%\label{fig:morphable}
%\bigskip
%\end{minipage}


