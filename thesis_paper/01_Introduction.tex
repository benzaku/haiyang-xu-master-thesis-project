\chapter{Introduction}
\label{chapter:introduction}
\TODO{Here is just a test for literature citation. \\
\cite{Krekhov:12:MSc}, \cite{Hoppe:1996:PM}, \cite{Hoppe:1997:VRP}, \cite{Kim:2001:trulyselective}, \cite{Kim:2003:TransitiveMeshSpace}, \cite{Kim:04:VDstreaming}, \cite{Yang:2004:VDMeshTrans}, \cite{Deb:2004:DesignStreamSys}
, \cite{Callahan:2006:PVR}, \cite{Pacanowski:2008:ESS}, \cite{Cheng:2007:AMP}, \cite{Bajaj:1999:PCTriMesh}, \cite{Khodakovsky:2000:PGC}, \cite{Deb:2006:RSRT}, \cite{Hoppe:1998:EIPM}
}
\\
\TODO{introduce this paper! }
\\

With the ever fast development of modern computer science, computer graphics and visualization has become a big topic. And with the more and more advanced 3D scanner and surface reconstruction technology, people are able to get extremely detailed 3D model from real objects such as sculptures. 
And mean while, in recent years, various mobile devices (such iPhone, iPad, Google Nexus series, etc.) with much powerful computing resource are being designed and manufactured. With faster CPU/GPU and larger memory, these hand-held devices are possible to run graphics programs and view 3D models. Therefore, how to view these large-scale 3D models becomes a hot topic. 

\smallskip

In this thesis, we proposed a streaming framework for large scale geometry models. In this framework users can connect our server from a mobile device (e.g. iPad) and view the 3D geometry model they choose progressively. User can browse the model using drag-and-zoom gesture on the multi-touch screen of the device and the system will refine the corresponding part of the viewing model according to users' viewing angle. Therefore our system can provide view-dependent, selective geometry streaming. 

\smallskip
In the following paragraphs of this chapter, we will introduce in detail the motivation of the thesis in Section~\ref{section:motivation}. And in Section~\ref{section:background} we will introduce the background of this topic. And Section~\ref{section:relWork} related works in the area of this topic will be listed and discussed. 

\smallskip
And then In Chapter~\ref{chapter:BasicConcepts}, we will describe some basic theoretical and technical concepts behind the topic of this thesis including \emph{Progressive Mesh}, the \emph{Quadric Error Metrics}, \emph{View-dependent Progressive Mesh}, \etc

\smallskip
In Chapter~\ref{chapter:SystemDesign}, we will describe the overall system architecture design including the server architecture and client architecture. 

\smallskip
In Chapter~\ref{chapter:SystemImplementation}, we will describe in detail about the implementation of our geometry streaming framework. In this chapter, both server and client side implementation will be illustrated. And after that there will be a short discussion of the implementation. 

\smallskip
In Chapter~\ref{chapter:ExperimentalEvaluation} and Chapter~\{}

\section{Motivation}
\label{section:motivation}
\TODO{In this section the motivation of our project will be described.}

\section{Background}
\label{section:background}
\TODO{Here we will introduce background information of this topic.}

\section{Related Work}
\label{section:relWork}
\TODO{In this section we will introduce previous research in this area. }

%Picture
%\noindent
%\begin{minipage}{\linewidth}
%\makebox[\linewidth]{%
%\includegraphics[width=1.0\textwidth]{images/morphable.pdf}}
%\captionof{figure}{MorphableUI generates user-tailored interfaces for arbitrary applications in arbitrary environments. Users are able to use all available devices to control as many applications as needed. User behavior is analyzed by the system to increase the user experience.}% only if needed
%\label{fig:morphable}
%\bigskip
%\end{minipage}


